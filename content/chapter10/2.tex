
path类在整个文件系统库中用于表示文件或目录路径。在符合posix的系统上,例如macOS和Linux,路径对象使用char类型来表示文件名。在Windows上,path使用wchar\_t。在Windows上,cout和format()不会显示wchar\_t字符的原始字符串。这意味着没有简单的方法来编写使用文件系统库,并且无法直接POSIX和Windows之间编写可移植的代码。

我们需要使用预处理器指令为Windows编写特定版本的代码。对于某些代码库来说,这可能是一个合理的解决方案,但对于本书来说,它很丑陋,不能满足简单、可移植、可重用的特性。

优雅的解决方案是为path类使用C++20的format进行格式化特化。这就可以简单地、可移植地显示path对象。

\subsubsection{How to do it…}

这个示例中,我们编写了一个格式化特化,用于fs::path类:

\begin{itemize}
\item 
方便起见,从命名空间别名开始。所有的文件系统名称都在std::filesystem命名空间中:

\begin{lstlisting}[style=styleCXX]
namespace fs = std::filesystem;
\end{lstlisting}

\item 
path类的格式化特化简单而简洁:

\begin{lstlisting}[style=styleCXX]
template<>
struct std::formatter<fs::path>:
std::formatter<std::string> {
	template<typename FormatContext>
	auto format(const fs::path& p, FormatContext& ctx) {
		return format_to(ctx.out(), "{}", p.string());
	}
};
\end{lstlisting}

这里,专为fs::path类型设置了格式,使用其string()方法来获得可输出的字符表示。我们不能使用c\_str()方法,因为其无法在Windows上处理wchar\_t字符。

本书的第1章中有关于格式化特化的更完整的解释。

\item 
在main()函数中,可以使用命令行传递文件名或路径:

\begin{lstlisting}[style=styleCXX]
int main(const int argc, const char** argv) {
	if(argc != 2) {
		fs::path fn{ argv[0] };
		cout << format("usage: {} <path>\n",
		fn.filename());
		return 0;
	}

	fs::path dir{ argv[1] };
	if(!fs::exists(dir)) {
		cout << format("path: {} does not exist\n",
		dir);
		return 1;
	}

	cout << format("path: {}\n", dir);
	cout << format("filename: {}\n", dir.filename());
	cout << format("cannonical: {}\n",
	fs::canonical(dir));
}
\end{lstlisting}

argc和argv参数是标准的命令行参数。

argv[0]始终是可执行文件本身的完整目录路径和文件名,若没有正确数量的参数,则显示argv[0]的文件名部分作为使用消息的一部分。

这个例子中,我们使用了一些文件系统函数:

\begin{itemize}
\item 
fs::exists()函数的作用是:检查目录或文件是否存在。

\item 
dir是一个路径对象。可以直接将它传递给format(),使用特化来显示路径的字符串表示形式。

\item 
filename()方法返回一个新的路径对象,直接将其传递给format()。

\item 
fs::cannonical()函数的作用是:接受一个路径对象,并返回一个带有规范绝对目录路径的新路径对象。我们将这个路径对象直接传递给format(),然后就会从cannical()返回的目录路径,并进行显示。
\end{itemize}

输出为:

\begin{tcblisting}{commandshell={}}
$ ./formatter ./formatter.cpp
path: ./formatter.cpp
filename: formatter.cpp
cannonical: /home/billw/working/chap10/formatter.cpp
\end{tcblisting}

\end{itemize}

\subsubsection{How it works…}

fs::path类在整个文件系统库中用于表示目录路径和文件名。通过提供格式化特化,可以轻松地跨平台一致地显示路径对象。

path类提供了一些有用的方法。可以遍历一个path对象,看看它的组成:

\begin{lstlisting}[style=styleCXX]
fs::path p{ "~/include/bwprint.h" };
cout << format("{}\n", p);
for(auto& x : p) cout << format("[{}] ", x);
cout << '\n';
\end{lstlisting}

输出为:

\begin{tcblisting}{commandshell={}}
~/include/bwprint.h
[~] [include] [bwprint.h]
\end{tcblisting}

迭代器为路径的每个元素返回一个path对象。

也可以得到路径的不同部分:

\begin{lstlisting}[style=styleCXX]
fs::path p{ "~/include/bwprint.h" };
cout << format("{}\n", p);
cout << format("{}\n", p.stem());
cout << format("{}\n", p.extension());
cout << format("{}\n", p.filename());
cout << format("{}\n", p.parent_path());
\end{lstlisting}

输出为:

\begin{tcblisting}{commandshell={}}
~/include/bwprint.h
bwprint
.h
bwprint.h
~/include
\end{tcblisting}

在本章中,我们将继续为path类使用这个格式化特化。



