

The filesystem library provides a directory\_entry class with directory-related information about a given path. We can use this to create useful directory listings.

\subsubsection{How to do it…}

In this recipe, we create a directory listing utility using the information in the directory\_entry class:

\begin{itemize}
\item 
We start with our namespace alias and formatter specialization for displaying path objects:

\begin{lstlisting}[style=styleCXX]
namespace fs = std::filesystem;
template<>
struct std::formatter<fs::path>:
std::formatter<std::string> {
	template<typename FormatContext>
	auto format(const fs::path& p, FormatContext& ctx) {
		return format_to(ctx.out(), "{}", p.string());
	}
};
\end{lstlisting}

\item 
The directory\_iterator class makes it easy to list a directory:

\begin{lstlisting}[style=styleCXX]
int main() {
	constexpr const char* fn{ "." };
	const fs::path fp{fn};
	for(const auto& de : fs::directory_iterator{fp}) {
		cout << format("{} ", de.path().filename());
	}
	cout << '\n';
}
\end{lstlisting}

Output:

\begin{tcblisting}{commandshell={}}
chrono Makefile include chrono.cpp working formatter
testdir formatter.cpp working.cpp
\end{tcblisting}

\item 
We can add command-line options to make this work, like Unix ls:

\begin{lstlisting}[style=styleCXX]
int main(const int argc, const char** argv) {
	fs::path fp{ argc > 1 ? argv[1] : "." };
	if(!fs::exists(fp)) {
		const auto cmdname {
			fs::path{argv[0]}.filename() };
		cout << format("{}: {} does not exist\n",
			cmdname, fp);
		return 1;
	}
	if(is_directory(fp)) {
		for(const auto& de :
		fs::directory_iterator{fp}) {
			cout << format("{} ",
			de.path().filename());
		}
	} else {
		cout << format("{} ", fp.filename());
	}
	cout << '\n';
}
\end{lstlisting}

If there is a command-line argument, we use it to create a path object. Otherwise, we use "." for the current directory.

We check if the path exists with if\_exists(). If not, we print an error message and exit. The error message includes cmdname from argv[0].

Next, we check is\_directory(). If we have a directory, we loop through a directory\_iterator for each entry. directory\_iterator iterates over directory\_entry objects. de.path().filename() gets the path and filename from each directory\_entry object.

Output:

\begin{tcblisting}{commandshell={}}
$ ./working
chrono Makefile include chrono.cpp working formatter
testdir formatter.cpp working.cpp
$ ./working working.cpp
working.cpp
$ ./working foo.bar
working: foo.bar does not exist
\end{tcblisting}

\item 
If we want our output sorted, we can store our directory\_entry objects in a sortable container.

Let's create an alias for fs::directory\_entry. We'll be using this a lot. This goes at the top of the file:

\begin{lstlisting}[style=styleCXX]
using de = fs::directory_entry;
\end{lstlisting}

At the top of main(), we declare a vector of de objects:

\begin{lstlisting}[style=styleCXX]
vector<de> entries{};
\end{lstlisting}

Inside the is\_directory() block, we load the vector, sort it, and then display it:

\begin{lstlisting}[style=styleCXX]
if(is_directory(fp)) {
	for(const auto& de : fs::directory_iterator{fp}) {
		entries.emplace_back(de);
	}
	std::sort(entries.begin(), entries.end());
	for(const auto& e : entries) {
		cout << format("{} ", e.path().filename());
	}
} else { ...
\end{lstlisting}

Now our output is sorted:

\begin{tcblisting}{commandshell={}}
Makefile chrono chrono.cpp formatter formatter.cpp
include testdir working working.cpp
\end{tcblisting}

Notice that Makefile is sorted first, apparently out of order. This is because capital letters sort before lowercase in ASCII order.

\item 
If we want a case-insensitive sort, we need a comparison function that ignores case. First, we need a function to return a lowercase string:

\begin{lstlisting}[style=styleCXX]
string strlower(string s) {
	auto char_lower = [](const char& c) -> char {
		if(c >= 'A' && c <= 'Z') return c + ('a' - 'A');
		else return c;
	};
	std::transform(s.begin(), s.end(), s.begin(),
		char_lower);
	return s;
}
\end{lstlisting}

Now we need a function that compares two directory\_entry objects, using strlower():

\begin{lstlisting}[style=styleCXX]
bool dircmp_lc(const de& lhs, const de& rhs) {
	const auto lhstr{ lhs.path().string() };
	const auto rhstr{ rhs.path().string() };
	return strlower(lhstr) < strlower(rhstr);
}
\end{lstlisting}

Now we can use dircmp\_lc() in our sort:

\begin{lstlisting}[style=styleCXX]
std::sort(entries.begin(), entries.end(), dircmp_lc);
\end{lstlisting}

Our output is now sorted ignoring case:

\begin{tcblisting}{commandshell={}}
chrono chrono.cpp formatter formatter.cpp include
Makefile testdir working working.cpp
\end{tcblisting}

\item 
At this point, we have a simple directory listing utility.

There's a lot more information available from the filesystem library. Let's create a print\_dir() function to gather more information and format it for display in the style of Unix ls:

\begin{lstlisting}[style=styleCXX]
void print_dir(const de& dir) {
	using fs::perms;
	const auto fpath{ dir.path() };
	const auto fstat{ dir.symlink_status() };
	const auto fperm{ fstat.permissions() };
	const uintmax_t fsize{
		is_regular_file(fstat) ? file_size(fpath) : 0 };
	const auto fn{ fpath.filename() };
	
	string suffix{};
	if(is_directory(fstat)) suffix = "/";
	else if((fperm & perms::owner_exec) != perms::none) {
		suffix = "*";
	}
	cout << format("{}{}\n", fn, suffix);
}
\end{lstlisting}

The print\_dir() function takes a directory\_entry argument. We then retrieve some useful objects from the directory\_entry object:

\begin{itemize}
\item 
dir.path() returns a path object

\item 
dir.symlink\_status() returns a file\_status object, without following symbolic links.

\item 
fstat.permissions() returns a perms object.

\item 
fsize is the size of the file and fn is the filename string. We'll look more closely at each of these as we use them.
\end{itemize}

Unix ls uses trailing characters, after the filename, to indicate a directory or an executable. We test the fstat object with is\_directory() to see if the file is a directory and add a trailing / to the filename. Likewise, we can test if a file is executable with the fperm object.

We call print\_dir() from main() in the for loop after sort():

\begin{lstlisting}[style=styleCXX]
std::sort(entries.begin(), entries.end(), dircmp_lc);
for(const auto& e : entries) {
	print_dir(e);
}
\end{lstlisting}

Our output now looks like this:

\begin{tcblisting}{commandshell={}}
chrono*
chrono.cpp
formatter*
formatter.cpp
include*
Makefile
testdir/
working*
working.cpp
\end{tcblisting}

\item 
Notice the include* entry. That's actually a symbolic link. Let's notate that properly by following the link to get the target path:

\begin{lstlisting}[style=styleCXX]
string suffix{};
if(is_symlink(fstat)) {
	suffix = " -> ";
	suffix += fs::read_symlink(fpath).string();
}
else if(is_directory(fstat)) suffix = "/";
else if((fperm & perms::owner_exec) != perms::none)
suffix = "*";
\end{lstlisting}

The read\_symlink() function returns a path object. We take the string() representation of the returned path object and add it to the suffix for this output:

\begin{tcblisting}{commandshell={}}
chrono*
chrono.cpp
formatter*
formatter.cpp
include -> /Users/billw/include
Makefile
testdir/
working*
working.cpp
\end{tcblisting}

\item 
The Unix ls command also includes a string of characters to indicate a file's permission bits. It looks something like this: drwxr-xr-x.

The first character indicates the type of the file, for example: d for directory, l for symbolic link, and - for a regular file.

The type\_char() function returns the appropriate character:

\begin{lstlisting}[style=styleCXX]
char type_char(const fs::file_status& fstat) {
		 if(is_symlink(fstat)) return 'l';
	else if(is_directory(fstat)) return 'd';
	else if(is_character_file(fstat)) return 'c';
	else if(is_block_file(fstat)) return 'b';
	else if(is_fifo(fstat)) return 'p';
	else if(is_socket(fstat)) return 's';
	else if(is_other(fstat)) return 'o';
	else if(is_regular_file(fstat)) return '-';
	return '?';
}
\end{lstlisting}

The rest of the string is in three triplets. Each triplet includes positions for the read, write, and execute permission bits, in the form rwx. If a bit is not set, its character is replaced by a -. There are three triplets for three sets of permissions: owner, group, and other, respectively.

\begin{lstlisting}[style=styleCXX]
string rwx(const fs::perms& p) {
	using fs::perms;
	auto bit2char = [&p](perms bit, char c) {
		return (p & bit) == perms::none ? '-' : c;
	};
	return { bit2char(perms::owner_read, 'r'),
		bit2char(perms::owner_write, 'w'),
		bit2char(perms::owner_exec, 'x'),
		bit2char(perms::group_read, 'r'),
		bit2char(perms::group_write, 'w'),
		bit2char(perms::group_exec, 'x'),
		bit2char(perms::others_read, 'r'),
		bit2char(perms::others_write, 'w'),
		bit2char(perms::others_exec, 'x') };
}
\end{lstlisting}

The perms object represents the POSIX permissions bitmap, but it's not necessarily implemented as bits. Each entry must be compared to the perms::none value.
Our lambda function fulfills this requirement.

We add this definition to the top of our print\_dir() function:

\begin{lstlisting}[style=styleCXX]
const auto permstr{ type_char(fstat) + rwx(fperm) };
\end{lstlisting}

We update our format() string:

\begin{lstlisting}[style=styleCXX]
cout << format("{} {}{}\n", permstr, fn, suffix);
\end{lstlisting}

And we get this output:

\begin{tcblisting}{commandshell={}}
-rwxr-xr-x chrono*
-rw-r--r-- chrono.cpp
-rwxr-xr-x formatter*
-rw-r--r-- formatter.cpp
lrwxr-xr-x include -> /Users/billw/include
-rw-r--r-- Makefile
drwxr-xr-x testdir/
-rwxr-xr-x working*
-rw-r--r-- working.cpp
\end{tcblisting}

\item 
Now, let's add a size string. The fsize value is from the file\_size() function, which returns a std::uintmax\_t type. This represents the maximum size natural integer on the target system. uintmax\_t is not always the same as size\_t and does not always convert easily. Notably, uintmax\_t is 32 bits on Windows, where size\_t is 64 bits:

\begin{lstlisting}[style=styleCXX]
string size_string(const uintmax_t fsize) {
	constexpr const uintmax_t kilo{ 1024 };
	constexpr const uintmax_t mega{ kilo * kilo };
	constexpr const uintmax_t giga{ mega * kilo };
	string s;
	if(fsize >= giga ) return
		format("{}{}", (fsize + giga / 2) / giga, 'G');
	else if (fsize >= mega) return
		format("{}{}", (fsize + mega / 2) / mega, 'M');
	else if (fsize >= kilo) return
		format("{}{}", (fsize + kilo / 2) / kilo, 'K');
	else return format("{}B", fsize);
}
\end{lstlisting}

I chose to use 1,024 as 1K in this function, as that appears to be the default on both Linux and BSD Unix. In production, this could be a command-line option.

We update our format() string in main():

\begin{lstlisting}[style=styleCXX]
cout << format("{} {:>6} {}{}\n",
	permstr, size_string(fsize), fn, suffix);
\end{lstlisting}

Now, we get this output:

\begin{tcblisting}{commandshell={}}
-rwxr-xr-x 284K chrono*
-rw-r--r--   2K chrono.cpp
-rwxr-xr-x 178K formatter*
-rw-r--r-- 906B formatter.cpp
lrwxr-xr-x   0B include -> /Users/billw/include
-rw-r--r-- 642B Makefile
drwxr-xr-x   0B testdir/
-rwxr-xr-x 197K working*
-rw-r--r--   5K working.cpp
\end{tcblisting}

\begin{tcolorbox}[colback=webgreen!5!white,colframe=webgreen!75!black,title=Note]
This utility is designed for POSIX systems, such as Linux and macOS. It works on a Windows system, but the Windows permissions system is different from the POSIX system. On Windows, the permissions bits will always appear fully set.
\end{tcolorbox}

\end{itemize}

\subsubsection{How it works…}

The filesystem library carries a rich set of information through its directory\_entry and related classes. The major classes we used in this recipe include:

\begin{itemize}
\item 
The path class represents a file system path, according to the rules of the target system. A path object is constructed from a string or another path. It need not represent an existing path, or even a possible path. The path string is parsed to component parts, including a root name, root directory, and an optional series of filenames and directory separators.

\item 
The directory\_entry class carries a path object as a member, and may also store additional attributes, including hard link count, status, symbolic link, file size, and last write time.

\item 
The file\_status class carries information about the type and permissions of a file. A perms object may be a member of file\_status, representing the permissions structure of a file.
\end{itemize}

There are two functions for retrieving a perms object from file\_status. The status() function and the symlink\_status() function both return a perms object. The difference is in how they handle a symbolic link. The status() function will follow a symbolic link and return the perms from the target file. symlink\_status() will return the perms from the symbolic link itself.

\subsubsection{There's more…}

I had intended to include the last-write time of each file in the directory listing.

The directory\_entry class has a member function, last\_write\_time(), which returns a file\_time\_type object representing the timestamp of the last time the file was written.

Unfortunately, at the time of writing, the available implementations lack a portable way to convert a file\_time\_type object to a standard chrono::sys\_time, suitable for use with cout or format().

For now, here's a solution that works with GCC:

\begin{lstlisting}[style=styleCXX]
string time_string(const fs::directory_entry& dir) {
	using std::chrono::file_clock;
	auto file_time{ dir.last_write_time() };
	return format("{:%F %T}",
		file_clock::to_sys(dir.last_write_time()));
}
\end{lstlisting}

It's recommended that user code should use std::chrono::clock\_cast instead of file::clock::to\_sys to convert time points between clocks. Unfortunately, none of the currently available implementations have a working std::chrono::clock\_cast specialization for this purpose.

Using this time\_string() function, we can add to print\_dir():

\begin{lstlisting}[style=styleCXX]
const string timestr{ time_string(dir) };
\end{lstlisting}

We can then change the format() string:

\begin{lstlisting}[style=styleCXX]
cout << format("{} {:>6} {} {}{}\n",
	permstr, sizestr, timestr, fn, suffix);
\end{lstlisting}

And we get this output:

\begin{tcblisting}{commandshell={}}
-rwxr-xr-x 248K 2022-03-09 09:39:49 chrono*
-rw-r--r--   2K 2022-03-09 09:33:56 chrono.cpp
-rwxr-xr-x 178K 2022-03-09 09:39:49 formatter*
-rw-r--r-- 906B 2022-03-09 09:33:56 formatter.cpp
lrwxrwxrwx   0B 2022-02-04 11:39:53 include -> /home/billw/
include
-rw-r--r-- 642B 2022-03-09 14:08:37 Makefile
drwxr-xr-x   0B 2022-03-09 10:38:39 testdir/
-rwxr-xr-x 197K 2022-03-12 17:13:46 working*
-rw-r--r--   5K 2022-03-12 17:13:40 working.cpp
\end{tcblisting}

This works on Debian with GCC-11. Do not expect it to work without modification on any other systems.










