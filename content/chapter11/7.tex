
在我的职业生涯中,使用过很多编程语言。学习一门新语言时,我喜欢做一个项目,让我接触到语言的细微差别。numwords类是我为此目的最喜欢的练习之一。多年来,我用数十种语言编写过它,包括多次用C和C++进行练习。

numwords是一个用单词拼写数字的类,可以用于银行和会计应用。使用起来像这样:

\begin{lstlisting}[style=styleCXX]
int main() {
	bw::numword nw{};
	uint64_t n;
	nw = 3; bw::print("n is {}, {}\n", nw.getnum(), nw);
	nw = 47; bw::print("n is {}, {}\n", nw.getnum(), nw);
	n = 100073; bw::print("n is {}, {}\n", n,
		bw::numword{n});
	n = 1000000001; bw::print("n is {}, {}\n", n,
		bw::numword{n});
	n = 123000000000; bw::print("n is {}, {}\n", n,
		bw::numword{n});
	n = 1474142398007; bw::print("n is {}, {}\n", n,
		nw.words(n));
	n = 999999999999999999; bw::print("n is {}, {}\n", n,
		nw.words(n));
	n = 1000000000000000000; bw::print("n is {}, {}\n", n,
		nw.words(n));
}
\end{lstlisting}

输出为:

\begin{tcblisting}{commandshell={}}
n is 3, three
n is 47, forty-seven
n is 100073, one hundred thousand seventy-three
n is 1000000001, one billion one
n is 123000000000, one hundred twenty-three billion
n is 1474142398007, one trillion four hundred seventy-four
billion one hundred forty-two million three hundred ninetyeight thousand seven
n is 999999999999999999, nine hundred ninety-nine quadrillion
nine hundred ninety-nine trillion nine hundred ninety-nine
billion nine hundred ninety-nine million nine hundred ninetynine thousand nine hundred ninety-nine
n is 1000000000000000000, error
\end{tcblisting}

\subsubsection{How to do it…}

这个示例起源于创建“生产就绪”的代码练习。因此,由三个不同的文件组成:

\begin{itemize}
\item 
numword.h是numwords类的头/接口文件。

\item 
numword.cpp是numwords类的实现文件。

\item 
numword-test.cpp是用于测试数字词类的应用程序文件。
\end{itemize}

这个类本身大约有180行代码。我们在这里只讨论重点代码段,读者们可以在这里找到完整的源代码,\url{https://github.com/PacktPublishing/CPP-20STL-Cookbook/tree/main/chap11/numword}。

\begin{itemize}
\item 
numword.h文件中,将类放在bw命名空间中,并以一些using语句开始:

\begin{lstlisting}[style=styleCXX]
namespace bw {
	using std::string;
	using std::string_view;
	using numnum = uint64_t;
	using bufstr = std::unique_ptr<string>;
\end{lstlisting}

整个代码中使用string和string\_view对象。

uint64\_t是我们的主要整数类型,因为这个类称为numword,所以我喜欢将numnum作为整数类型。

\_bufstr是主要的输出缓冲区,一个包装在unique\_ptr中的字符串,遵从RAII的内存管理。

\item 
也有一些用于不同目的的常量:

\begin{lstlisting}[style=styleCXX]
constexpr numnum maxnum = 999'999'999'999'999'999;
constexpr int zero_i{ 0 };
constexpr int five_i{ 5 };
constexpr numnum zero{ 0 };
constexpr numnum ten{ 10 };
constexpr numnum twenty{ 20 };
constexpr numnum hundred{ 100 };
constexpr numnum thousand{ 1000 };
\end{lstlisting}

maxnum常数对于大多数目的应该是足够的,其余部分用于避免在代码中使用文字。

\item 
主要的数据结构是string\_view对象的constexpr数组,表示输出中使用的单词。因为其提供了最小开销的封装,所以string\_view类非常适合这些常量:

\begin{lstlisting}[style=styleCXX]
constexpr string_view errnum{ "error" };
constexpr string_view _singles[] {
	"zero", "one", "two", "three", "four", "five",
	"six", "seven", "eight", "nine"
};
constexpr string_view _teens[] {
	"ten", "eleven", "twelve", "thirteen", "fourteen",
	"fifteen", "sixteen", "seventeen", "eighteen",
	"nineteen"
};
constexpr string_view _tens[] {
	errnum, errnum, "twenty", "thirty", "forty",
	"fifty", "sixty", "seventy", "eighty", "ninety",
};
constexpr string_view _hundred_string = "hundred";
constexpr string_view _powers[] {
	errnum, "thousand", "million", "billion",
	"trillion", "quadrillion"
};
\end{lstlisting}

这些单词可以分成几个部分,在将数字转换成单词时很有用。许多语言使用类似的分解,所以这个结构应该很适用于语言翻译。

\item 
numword类有几个私有成员:

\begin{lstlisting}[style=styleCXX]
class numword {
	bufstr _buf{ std::make_unique<string>(string{}) };
	numnum _num{};
	bool _hyphen_flag{ false };
\end{lstlisting}

\begin{itemize}
\item 
\_buf是输出字符串缓冲区,内存由unique\_ptr管理。

\item 
\_num保存当前数值。

\item 
\_hyphen\_flag在翻译过程中,用于在单词之间插入连字符,而不是空格字符。
\end{itemize}

\item 
这些私有方法用于操作输出缓冲区。

\begin{lstlisting}[style=styleCXX]
void clearbuf();
size_t bufsize();
void appendbuf(const string& s);
void appendbuf(const string_view& s);
void appendbuf(const char c);
void appendspace();
\end{lstlisting}

还有一个pow\_i()私有方法用于计算$x^y$:

\begin{lstlisting}[style=styleCXX]
numnum pow_i(const numnum n, const numnum p);
\end{lstlisting}

pow\_i()用于区分字输出的数字值的各个部分。

\item 
公共接口包括构造函数和调用words()方法的各种方法,该方法完成将numnum转换为单词字符串的工作:

\begin{lstlisting}[style=styleCXX]
numword(const numnum& num = 0) : _num(num) {}
numword(const numword& nw) : _num(nw.getnum()) {}
const char * version() const { return _version; }
void setnum(const numnum& num) { _num = num; }
numnum getnum() const { return _num; }
numnum operator= (const numnum& num);
const string& words();
const string& words(const numnum& num);
const string& operator() (const numnum& num) {
	return words(num); };
\end{lstlisting}

\item 
实现文件numword.cpp中,大部分工作在words()成员函数中处理:

\begin{lstlisting}[style=styleCXX]
const string& numword::words( const numnum& num ) {
	numnum n{ num };
	clearbuf();
	if(n > maxnum) {
		appendbuf(errnum);
		return *_buf;
	}
	if (n == 0) {
		appendbuf(_singles[n]);
		return *_buf;
	}
	// powers of 1000
	if (n >= thousand) {
		for(int i{ five_i }; i > zero_i; --i) {
			numnum power{ pow_i(thousand, i) };
			numnum _n{ ( n - ( n % power ) ) / power };
			if (_n) {
				int index = i;
				numword _nw{ _n };
				appendbuf(_nw.words());
				appendbuf(_powers[index]);
				n -= _n * power;
			}
		}
	}
	// hundreds
	if (n >= hundred && n < thousand) {
		numnum _n{ ( n - ( n % hundred ) ) / hundred };
		numword _nw{ _n };
		appendbuf(_nw.words());
		appendbuf(_hundred_string);
		n -= _n * hundred;
	}
	// tens
	if (n >= twenty && n < hundred) {
		numnum _n{ ( n - ( n % ten ) ) / ten };
		appendbuf(_tens[_n]);
		n -= _n * ten;
		_hyphen_flag = true;
	}
	// teens
	if (n >= ten && n < twenty) {
		appendbuf(_teens[n - ten]);
		n = zero;
	}
	// singles
	if (n > zero && n < ten) {
		appendbuf(_singles[n]);
	}
	return *_buf;
}
\end{lstlisting}

该函数的每个部分以10的次幂的模量剥离数字的一部分,在千万的情况下递归,并从string\_view常量数组中追加字符串。

\item 
appendbuf()有三个重载。一个添加字符串:

\begin{lstlisting}[style=styleCXX]
void numword::appendbuf(const string& s) {
	appendspace();
	_buf->append(s);
}
\end{lstlisting}

另一个添加string\_view:

\begin{lstlisting}[style=styleCXX]
void numword::appendbuf(const string_view& s) {
	appendspace();
	_buf->append(s.data());
}
\end{lstlisting}

第三个添加单个字符:

\begin{lstlisting}[style=styleCXX]
void numword::appendbuf(const char c) {
	_buf->append(1, c);
}
\end{lstlisting}

appendspace()方法根据上下文附加一个空格字符或连字符:

\begin{lstlisting}[style=styleCXX]
void numword::appendspace() {
	if(bufsize()) {
		appendbuf( _hyphen_flag ? _hyphen : _space);
		_hyphen_flag = false;
	}
}
\end{lstlisting}

\item 
numword-test.cpp文件是bw::numword的测试环境,包括格式化特化:

\begin{lstlisting}[style=styleCXX]
template<>
struct std::formatter<bw::numword>:
std::formatter<unsigned> {
	template<typename FormatContext>
	auto format(const bw::numword& nw,
	FormatContext& ctx) {
		bw::numword _nw{nw};
		return format_to(ctx.out(), "{}",
		nw.words());
	}
};
\end{lstlisting}

可以直接将bw::numword对象传递给format()。

\item 
还有一个print()函数,绕过cout和iostream,直接将格式化输出发送到stdout:

\begin{lstlisting}[style=styleCXX]
namespace bw {
	template<typename... Args> constexpr void print(
	const std::string_view str_fmt, Args&&...
	args) {
		fputs(std::vformat(str_fmt,
		std::make_format_args(args...)).c_str(),
		stdout);
	}
};
\end{lstlisting}

这就可以使用print("\{\}\verb|\|n", nw),而不是通过cout管道。C++23标准中会有这样一个函数。

\item 
main()中,声明了一个bw::numword对象和用于测试的uint64\_t:

\begin{lstlisting}[style=styleCXX]
int main() {
	bw::numword nw{};
	uint64_t n{};
	
	bw::print("n is {}, {}\n", nw.getnum(), nw);
	...
\end{lstlisting}

numword对象初始化为0,从print()语句中得到如下输出:

\begin{tcblisting}{commandshell={}}
n is 0, zero
\end{tcblisting}

\item 
我们测试了多种调用numword的方法:

\begin{lstlisting}[style=styleCXX]
nw = 3; bw::print("n is {}, {}\n", nw.getnum(), nw);
nw = 47; bw::print("n is {}, {}\n", nw.getnum(), nw);
...
n = 100073; bw::print("n is {}, {}\n", n,
bw::numword{n});
n = 1000000001; bw::print("n is {}, {}\n", n,
bw::numword{n});
...
n = 474142398123; bw::print("n is {}, {}\n", n, nw(n));
n = 1474142398007; bw::print("n is {}, {}\n", n, nw(n));
...
n = 999999999999999999; bw::print("n is {}, {}\n", n,
nw(n));
n = 1000000000000000000; bw::print("n is {}, {}\n", n,
nw(n));
\end{lstlisting}

输出为:

\begin{tcblisting}{commandshell={}}
n is 3, three
n is 47, forty-seven
...
n is 100073, one hundred thousand seventy-three
n is 1000000001, one billion one
...
n is 474142398123, four hundred seventy-four billion
one hundred forty-two million three hundred ninety-eight
thousand one hundred twenty-three
n is 1474142398007, one trillion four hundred seventyfour billion one hundred forty-two million three hundred
ninety-eight thousand seven
...
n is 999999999999999999, nine hundred ninety-nine
quadrillion nine hundred ninety-nine trillion nine
hundred ninety-nine billion nine hundred ninety-nine
million nine hundred ninety-nine thousand nine hundred
ninety-nine
n is 1000000000000000000, error
\end{tcblisting}
\end{itemize}

\subsubsection{How it works…}

这个类主要是由数据结构驱动的。通过将string\_view对象组织为数组,可以轻松地将标量值转换为相应的单词:

\begin{lstlisting}[style=styleCXX]
appendbuf(_tens[_n]); // e.g., _tens[5] = "fifty"
\end{lstlisting}

剩下的就是数学问题了:

\begin{lstlisting}[style=styleCXX]
numnum power{ pow_i(thousand, i) };
numnum _n{ ( n - ( n % power ) ) / power };
if (_n) {
	int index = i;
	numword _nw{ _n };
	appendbuf(_nw.words());
	appendbuf(_powers[index]);
	n -= _n * power;
}
\end{lstlisting}

\subsubsection{There's more…}

我还有一个程序,使用numwords类以文字形式展示时间。其输出如下所示:

\begin{tcblisting}{commandshell={}}
$ ./saytime
three past five
\end{tcblisting}

测试模式下,输出如下所示:

\begin{tcblisting}{commandshell={}}
$ ./saytime test
00:00 midnight
00:01 one past midnight
11:00 eleven o'clock
12:00 noon
13:00 one o'clock
12:29 twenty-nine past noon
12:30 half past noon
12:31 twenty-nine til one
12:15 quarter past noon
12:30 half past noon
12:45 quarter til one
11:59 one til noon
23:15 quarter past eleven
23:59 one til midnight
12:59 one til one
13:59 one til two
01:60 OOR
24:00 OOR
\end{tcblisting}

我把其的实现留给读者们,作为本书最后的家庭作业。

















