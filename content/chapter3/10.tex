
set容器是关联容器,其中每个元素都是一个单独的值作为键。set中的元素按序排部,不允许重复。

set比一般的容器(如vector和map)有更少和更具体的用途。set的一个常见用途是从一组值中筛选重复项。

\subsubsection{How to do it…}

这个示例中,我们将从标准输入中读取单词,并过滤掉重复的单词。

\begin{itemize}
\item 
从定义istream迭代器的别名开始。我们将使用它从命令行获取输入。

\begin{lstlisting}[style=styleCXX]
using input_it = istream_iterator<string>;
\end{lstlisting}

\item 
main()函数中,将为单词定义一个set:

\begin{lstlisting}[style=styleCXX]
int main() {
	set<string> words;
\end{lstlisting}

set定义为一组字符串元素。

\item 
定义了用于inserter()函数的迭代器:

\begin{lstlisting}[style=styleCXX]
input_it it{ cin };
input_it end{};
\end{lstlisting}

结束迭代器使用其默认构造函数初始化,这称为流结束迭代器。当输入结束时,这个迭代器将与等价于cin迭代器。

\item 
inserter()函数用于将元素插入到set容器中:

\begin{lstlisting}[style=styleCXX]
copy(it, end, inserter(words, words.end()));
\end{lstlisting}

我们使用std::copy()方便地从输入流中复制单词。

\item 
现在可以打印集合来查看结果:

\begin{lstlisting}[style=styleCXX]
for(const string & w : words) {
	cout << format("{} ", w);
}
cout << '\n';
\end{lstlisting}

\item 
可以通过将一堆单词作为输入来运行程序:

\begin{tcblisting}{commandshell={}}
$ echo "a a a b c this that this foo foo foo" | ./
set-words
a b c foo that this
\end{tcblisting}
\end{itemize}

该集合消除了重复项,并保留了插入的单词的排序列表。

\subsubsection{How it works…}

set容器是这示例的核心,其只保存唯一的元素。当插入重复元素时,该插入将失败。所以,最终将得到了一个由每个唯一元素组成的有序列表。

但这并不是这个示例唯一有趣的地方。

istream\_iterator是一个从流中读取对象的输入迭代器,可以像这样实例化输入迭代器:

\begin{lstlisting}[style=styleCXX]
istream_iterator<string> it{ cin };
\end{lstlisting}

现在有了一个来自cin流的string类型的输入迭代器。每次解引用此迭代器时,都会从输入流中返回一个单词。

我们还实例化了另一个istream\_iterator:

\begin{lstlisting}[style=styleCXX]
istream_iterator<string> end{};
\end{lstlisting}

这将调用默认构造函数,提供了一个特殊的流结束迭代器。当输入迭代器到达流的末尾时,等于流的结束迭代器。这对于结束循环很方便,比如copy()算法创建的循环。

copy()算法接受三个迭代器,要复制的范围的开始和结束,以及一个目标迭代器:

\begin{lstlisting}[style=styleCXX]
copy(it, end, inserter(words, words.end()));
\end{lstlisting}

inserter()函数的作用是:接受一个容器和插入点的迭代器,并返回容器及其元素的适当类型的insert\_iterator。

copy()和inserter()的组合使得将元素从流复制到set容器变得更容易。

