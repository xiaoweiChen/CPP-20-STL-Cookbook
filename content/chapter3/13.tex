
对于作者来说,确保使用了不同的句子长度,或者确保句子不太长很有帮助。现在,来构建一个评估文本句子长度的工具。在使用STL时,选择适当的容器是关键。若需要一些有序的东西,通常最好使用关联容器,比如map或multimap。然而,在这种情况下,因为需要自定义排序,所以vector可能更合适。

vector容器通常是STL容器中最灵活的。每当另一种容器类型看起来合适,但缺少一个重要功能时,vector通常是一种有效的解决方案。所以,我们需要一个自定义排序时,就很适合使用vector完成。

这个示例使用了vector的vector。内部vector存储句子中的单词,外部vector存储内部vector,这在保留所有相关数据的同时提供了很大的灵活性。

\subsubsection{How to do it…}

程序需要读入单词,找到句子的结尾,存储和排序句子,然后打印出结果。

\begin{itemize}
\item 
先写一个小函数来判断什么时候到达了句尾:

\begin{lstlisting}[style=styleCXX]
bool is_eos(const string_view & str) {
	constexpr const char * end_punct{ ".!?" };
	for(auto c : str) {
		if(strchr(end_punct, c) != nullptr) return
		true;
	}
	return false;
}
\end{lstlisting}

is\_eos()函数使用string\_view,因为它很有效,而且不需要其他东西。然后,使用strchr()库函数检查单词是否包含句尾标点字符(".!?")。这是英语中结束句子最有可能的三个字符。

\item 
main()函数中,我们首先定义vector的vector:

\begin{lstlisting}[style=styleCXX]
vector<vector<string>> vv_sentences{vector<string>{}};
\end{lstlisting}

这定义了一个名为vv\_sentence的元素vector,类型为vector<string>。vv\_sentence对象初始化为第一个句子的空vector。

这将创建一个包含其他vector的vector,内部vector将每个包含一个词的句子。

\item 
现在可以处理单词流了:

\begin{lstlisting}[style=styleCXX]
for(string s{}; cin >> s; ) {
	vv_sentences.back().emplace_back(s);
	if(is_eos(s)) {
		vv_sentences.emplace_back(vector<string>{});
	}
}
\end{lstlisting}

for循环每次从输入流返回一个单词。vv\_sentence对象上的back()方法用于访问单词的当前vector,并且使用emplace\_back()添加当前单词。然后调用is\_eos()来查看这是否是句子的结尾。若是这样,可以添加一个新的空vector到vv\_sentence开始下一个句子。

\item 
我们总是在vv\_sentence的每个句尾字符之后添加一个新的空vector,所以需要在结尾得到一个空的句子vector。在这里进行检查,并在必要时将其删除:

\begin{lstlisting}[style=styleCXX]
	// delete back if empty
	if(vv_sentences.back().empty())
		vv_sentences.pop_back();
\end{lstlisting}

\item 
现在可以根据句子的大小对vv\_sentence vector进行排序:

\begin{lstlisting}[style=styleCXX]
	sort(vv_sentences, [](const auto& l,
		const auto& r) {
			return l.size() > r.size();
		});
\end{lstlisting}

这就是为什么vector对这个项目如此方便。使用ranges::sort()算法和一个简单的谓词,会使按大小降序排序非常快速和容易。

\item 
现在可以打印结果了:

\begin{lstlisting}[style=styleCXX]
	constexpr int WLIMIT{10};
	for(auto& v : vv_sentences) {
		size_t size = v.size();
		size_t limit{WLIMIT};
		cout << format("{}: ", size);
		for(auto& s : v) {
			cout << format("{} ", s);
			if(--limit == 0) {
				if(size > WLIMIT) cout << "...";
				break;
			}
		}
		cout << '\n';
	}
	cout << '\n';
}
\end{lstlisting}

外环和内环分别对应于外vector和内vector。简单地遍历这些vector,并使用format("{}: ", size)打印出内部vector的大小,然后使用format("{} ", s)打印出每个单词。若不想完整地打印非常长的句子,可以定义了10个单词的限制,若有更多,则打印省略号。

\item 
输出看起来像这样:

\begin{tcblisting}{commandshell={}}
$ ./sentences < sentences.txt
27: It can be useful for a writer to make sure ...
19: Whenever another container type seems appropriate,
but is missing one ...
18: If you need something ordered, it's often best to use
...
17: The inner vector stores the words of a sentence, and
...
16: In this case, however, since we need a descending
sort, ...
16: In this case, where we need our output sorted in ...
15: As you'll see, this affords a lot of flexibility
while ...
12: Let's build a tool that evaluates a text file for ...
11: The vector is generally the most flexible of the STL
...
9: Choosing the appropriate container key when using the
STL.
7: This recipe uses a vector of vectors.
\end{tcblisting}

\end{itemize}


\subsubsection{How it works…}

使用C标准库中的strchr()函数查找标点符号非常简单,所有的C及其标准库都包含在C++语言的定义中,可以在适当的时候使用。

\begin{lstlisting}[style=styleCXX]
bool is_eos(const string_view & str) {
	constexpr const char * end_punct{ ".!?" };
	for(auto c : str) {
		if(strchr(end_punct, c) != nullptr) return true;
	}
	return false;
}
\end{lstlisting}

若单词中间有标点符号,这个函数将无法正确地分开句子。这可能发生在某些形式的诗歌或格式糟糕的文本文件中。我曾见过用std::string迭代器和正则表达式来实现这一点,但对于我们来说,快速和简单就够了。

使用cin逐字读取文本文件:

\begin{lstlisting}[style=styleCXX]
for(string s{}; cin >> s; ) {
	...
}
\end{lstlisting}

这避免了一次性将一个大文件读入内存的开销。vector已经很大,包含文件的所有单词,没有必要将整个文本文件也保存在内存中。在文件过大的情况下,有必要找到另一种策略,或使用数据库。

“vector的vector”乍一看可能很复杂,但它并不比使用两个独立的vector复杂。

\begin{lstlisting}[style=styleCXX]
vector<vector<string>> vv_sentences{vector<string>{}};
\end{lstlisting}

这声明了一个外部vector,内部元素类型为vector<string>。外层vector命名为vv\_sentence。内vector是匿名的,不需要命名。这里的定义用一个空vector<string>对象初始化vv\_sentence对象。

当前的内部vector,可以作为vv\_sentence.back()使用:

\begin{lstlisting}[style=styleCXX]
vv_sentences.back().emplace_back(s);
\end{lstlisting}

当完成一个内部vector量时,可以简单地创建一个新vector:

\begin{lstlisting}[style=styleCXX]
vv_sentences.emplace_back(vector<string>{});
\end{lstlisting}

这将创建一个新的匿名vector<string>对象,并将其放置在vv\_sentence对象的后面。

