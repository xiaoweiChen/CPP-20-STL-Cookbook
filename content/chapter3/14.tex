
有序任务列表(或待办事项列表)是一种常见的计算应用程序,它是一个与优先级相关的任务列表,按反向数字顺序排序。

读者可能倾向于使用priority\_queue,因为它已经按照优先级(反向数字)顺序排序了。priority\_queue的缺点是没有迭代器,因此若不从队列中推入或弹出项,就很难对其进行操作。

对于本节的示例,我们将为有序列表使用一个multimap。multimap关联容器保持项目的顺序,并且可以使用反向迭代器以适当的排序顺序对其进行访问。

\subsubsection{How to do it…}

这是一个简短而简单的示例,用于初始化multimap并以相反的顺序打印。

\begin{itemize}
\item 
从multimap的类型别名开始:

\begin{lstlisting}[style=styleCXX]
using todomap = multimap<int, string>;
\end{lstlisting}

todomap是一个具有int键和字符串的multimap。

\item 
这里有一个小函数来倒序打印todomap:

\begin{lstlisting}[style=styleCXX]
void rprint(todomap& todo) {
	for(auto it = todo.rbegin(); it != todo.rend();
	++it) {
		cout << format("{}: {}\n", it->first,
		it->second);
	}
	cout << '\n';
}
\end{lstlisting}

可以使用反向迭代器来打印todomap。

\item 
main()函数很简单:

\begin{lstlisting}[style=styleCXX]
int main()
{
	todomap todo {
		{1, "wash dishes"},
		{0, "watch teevee"},
		{2, "do homework"},
		{0, "read comics"}
	};
	rprint(todo);
}
\end{lstlisting}

我们用任务初始化todomap。请注意,任务没有任何特定的顺序,但在键中确实有优先级。rprint()函数可以按优先级顺序打印它们。

\item 
输出如下所示:

\begin{tcblisting}{commandshell={}}
$ ./todo
2: do homework
1: wash dishes
0: read comics
0: watch teevee
\end{tcblisting}

待办事项列表会按优先级顺序打印出来。

\end{itemize}


\subsubsection{How it works…}

这是一个简短而简单的示例,使用multimap容器来保存优先级列表中的项。

需要一些说明的仅是rprint()函数:

\begin{lstlisting}[style=styleCXX]
void rprint(todomap& todo) {
	for(auto it = todo.rbegin(); it != todo.rend(); ++it) {
		cout << format("{}: {}\n", it->first, it->second);
	}
	cout << '\n';
}
\end{lstlisting}

注意反向迭代器rbegin()和rend()。不可能改变multimap的排序顺序,但是它提供了反向迭代器。这使得multimap的行为完全符合我们对优先级列表的要求。




