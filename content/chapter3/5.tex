
vector是STL中使用最广泛的容器之一,它和数组一样方便,但更强大和灵活。通常使用[]操作符来访问vector中的元素,如下所示:

\begin{lstlisting}[style=styleCXX]
vector v{ 19, 71, 47, 192, 4004 };
auto & i = v[2];
\end{lstlisting}

vector类还提供了一个成员函数,用于同样的目的:

\begin{lstlisting}[style=styleCXX]
auto & i = v.at(2);
\end{lstlisting}

结果一样,但有一个重要的区别。at()函数执行边界检查,而[]操作符不检查,[]操作符为了保持与原始C数组的兼容性。

\subsubsection{How to do it…}

有两种方法访问vector中带有索引的元素。at()成员函数执行边界检查,而[]操作符不检查。

\begin{itemize}
\item 
下面是一个简单的main()函数,其初始化一个vector并访问一个元素:

\begin{lstlisting}[style=styleCXX]
int main() {
	vector v{ 19, 71, 47, 192, 4004 };
	auto & i = v[2];
	cout << format("element is {}\n", i);
}
\end{lstlisting}

输出:

\begin{tcblisting}{commandshell={}}
element is 47
\end{tcblisting}

这里,我使用[]操作符直接访问vector中的第三个元素。与C++中的大多数顺序对象一样,索引从0开始,因此第三个元素是2。

\item 
这个vector有5个元素,从0到4。如果我试图访问5号元素,将超出vector的边界:

\begin{lstlisting}[style=styleCXX]
vector v{ 19, 71, 47, 192, 4004 };
auto & i = v[5];
cout << format("element is {}\n", i);
\end{lstlisting}

输出:

\begin{tcblisting}{commandshell={}}
element is 0
\end{tcblisting}

这个结果极具欺骗性。这是一个常见的错误,因为人类倾向于从1开始计数,而不是0。但是不能保证vector范围外的元素的值。

\item 
更糟糕的是,[]操作符会无声地允许对超出vector结尾的位置进行写入:

\begin{lstlisting}[style=styleCXX]
vector v{ 19, 71, 47, 192, 4004 };
v[5] = 2001;
auto & i = v[5];
cout << format("element is {}\n", i);
\end{lstlisting}

输出:

\begin{tcblisting}{commandshell={}}
element is 2001
\end{tcblisting}

现在已经写入内存,编译器已经默默地允许它,没有错误消息或崩溃。但是不要被骗了——这是极其危险的代码,在将来的某个时候会导致问题。越界内存访问是安全漏洞的主要原因之一。

\item 
解决方案是尽可能使用at()成员函数,而不是[]操作符:

\begin{lstlisting}[style=styleCXX]
vector v{ 19, 71, 47, 192, 4004 };
auto & i = v.at(5);
cout << format("element is {}\n", i);
\end{lstlisting}

现在出现了一个运行时异常:

\begin{tcblisting}{commandshell={}}
terminate called after throwing an instance of 'std::out_
of_range'
	what(): vector::_M_range_check: __n (which is 5) >=
this->size() (which is 5)
Aborted
\end{tcblisting}

代码编译时没有错误,但是at()函数检查容器的边界,并尝试访问这些边界之外的内存时,就抛出运行时异常。这是来自GCC编译器编译的代码的异常消息。在不同的环境中,信息也会不同。
\end{itemize}

\subsubsection{How it works…}

[]操作符和at()成员函数做同样的工作,回根据容器元素的索引位置提供对容器元素的直接访问。[]操作符不进行边界检查,因此在一些频繁迭代的应用程序中,会稍微快一点。

也就是说,at()函数应该是首选。虽然边界检查可能需要几个CPU周期,但这是一种成本很低的保险。对于大多数应用来说,这样做物有所值。

vector类通常用作直接访问容器,而array和deque容器也同时支持[]操作符和at()成员函数。这里的警告同样适用于他们。

\subsubsection{There's more…}

某些应用程序中,可能不希望应用程序在遇到出界条件时崩溃。这种情况下,可以捕获异常:

\begin{lstlisting}[style=styleCXX]
int main() {
	vector v{ 19, 71, 47, 192, 4004 };
	try {
		v.at(5) = 2001;
	} catch (const std::out_of_range & e) {
		std::cout <<
		format("Ouch!\n{}\n", e.what());
	}
	cout << format("end element is {}\n", v.back());
}
\end{lstlisting}

输出:

\begin{tcblisting}{commandshell={}}
Ouch!
vector::_M_range_check: __n (which is 5) >= this->size() (which is 5)
end element is 4004
\end{tcblisting}

try块捕获catch子句中指定的异常,当前的异常是std::out\_of\_range。e.what()函数返回一个C字串,其中包含来自STL库的错误消息。每个库都会有不同的消息。

记住,这也适用于array和deque容器。











