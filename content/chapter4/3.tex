
本节示例描述了一个简单的类,其生成一个可迭代范围,适合与基于范围的for循环一起使用。其思想是创建一个序列生成器,从开始值迭代到结束值。

为了完成这个任务,我们需要一个迭代器类,以及对象接口类。

\subsubsection{How to do it…}

这个示例有两个主要部分:主接口、Seq和迭代器类。

\begin{itemize}
\item 
首先,定义Seq类,只需要实现begin()和end()成员函数:

\begin{lstlisting}[style=styleCXX]
template<typename T>
class Seq {
	T start_{};
	T end_{};
public:
	Seq(T start, T end) : start_{start}, end_{end} {}
	iterator<T> begin() const {
		return iterator{start_};
	}
	iterator<T> end() const { return iterator{end_}; }
};
\end{lstlisting}

构造函数设置start\_和end\_变量,用于构造begin()和end()迭代器。成员函数begin()和end()返回迭代器对象。

\item 
迭代器类通常定义在容器类的公共部分中,称为成员类或嵌套类。我们将在Seq构造函数后面插入它:

\begin{lstlisting}[style=styleCXX]
public:
	Seq(T start, T end) : start_{ start }, end_{ end } {}
	class iterator {
		T value_{};
		public:
		explicit iterator(T position = 0)
		: value_{position} {}
		T operator*() const { return value_; }
		iterator& operator++() {
			++value_;
			return *this;
		}
		bool operator!=(const iterator& other) const {
			return value_ != other.value_;
		}
	};
\end{lstlisting}

迭代器类通常命名为iterator,其类型为Seq<type>::iterator。

迭代器构造函数是限定显式的,以避免隐式转换。

value\_变量由迭代器维护,可以解引用指针返回一个值。

支持基于范围的for循环的最低要求是解引用操作符、前自增操作符和不等比较操作符。

\item 
现在可以写一个main()函数来测试序列生成器:

\begin{lstlisting}[style=styleCXX]
int main()
{
	Seq<int> r{ 100, 110 };
	for (auto v : r) {
		cout << format("{} ", v);
	}
	cout << '\n';
}
\end{lstlisting}

这将构造一个Seq对象,并打印其序列。

输出如下所示:

\begin{tcblisting}{commandshell={}}
$ ./seq
100 101 102 103 104 105 106 107 108 109
\end{tcblisting}

\end{itemize}

\subsubsection{How it works…}

这个示例的重点是制作一个序列生成器,与基于范围的for循环一起工作。我们首先考虑基于范围的for循环的等效代码:

\begin{lstlisting}[style=styleCXX]
{
	auto begin_it{ std::begin(container) };
	auto end_it{ std::end(container) };
	for ( ; begin_it != end_it; ++begin_it) {
		auto v{ *begin_it };
		cout << v << '\n';
	}
}
\end{lstlisting}

从这段等价的代码中,可以推出使用for循环的要求:

\begin{itemize}
\item 
具有begin()和end()迭代器

\item 
迭代器支持不相等的比较操作符

\item 
迭代器支持前缀自增运算符

\item 
迭代器支持解引用操作符
\end{itemize}

主Seq类接口只有三个公共成员函数:构造函数、begin()和end()迭代器:

\begin{lstlisting}[style=styleCXX]
Seq(T start, T end) : start_{ start }, end_{ end } {}
iterator begin() const { return iterator{start_}; }
iterator end() const { return iterator{end_}; }
\end{lstlisting}

Seq::iterator类的实现携带了实际的负载:

\begin{lstlisting}[style=styleCXX]
class iterator {
	T value_{};
\end{lstlisting}

这是常见的配置,有效负载只能通过迭代器访问。

这里需要实现三个操作符:

\begin{lstlisting}[style=styleCXX]
T operator*() const { return value_; }
iterator& operator++() {
	++value_;
	return *this;
}
bool operator!=(const iterator& other) const {
	return value_ != other.value_;
}
\end{lstlisting}

就是基于范围的for循环所需的:

\begin{lstlisting}[style=styleCXX]
Seq<int> r{ 100, 110 };
for (auto v : r) {
	cout << format("{} ", v);
}
\end{lstlisting}

\subsubsection{There's more…}

通常,会将迭代器定义为容器的成员类,但不是必需的。这允许迭代器类型从属于容器类型:

\begin{lstlisting}[style=styleCXX]
Seq<int>::iterator it = r.begin();
\end{lstlisting}

因为auto类型,C++11之后具体类型就不那么重要了,但它仍然是最佳实践。





