
lambda本质上是一个匿名函数作为字面表达式:

\begin{lstlisting}[style=styleCXX]
auto la = []{ return "Hello\n"; };
\end{lstlisting}

变量la现在可以像函数一样使用:

\begin{lstlisting}[style=styleCXX]
cout << la();
\end{lstlisting}

可以传递给另一个函数:

\begin{lstlisting}[style=styleCXX]
f(la);
\end{lstlisting}

也可以传递给另一个lambda:

\begin{lstlisting}[style=styleCXX]
const auto la = []{ return "Hello\n"; };
const auto lb = [](auto a){ return a(); };
cout << lb(la);
\end{lstlisting}

输出:

\begin{tcblisting}{commandshell={}}
Hello
\end{tcblisting}

或者它可以匿名传递(作为一个文字):

\begin{lstlisting}[style=styleCXX]
const auto lb = [](auto a){ return a(); };
cout << lb([]{ return "Hello\n"; });
\end{lstlisting}

\subsubsection{闭包}

闭包这个术语通常应用于任何匿名函数。严格地说,闭包是一个允许在自己的词法范围之外使用符号的函数。

可能已经注意到lambda定义中的方括号:

\begin{lstlisting}[style=styleCXX]
auto la = []{ return "Hello\n"; };
\end{lstlisting}

方括号用于指定捕获列表,捕获是可从lambda体范围内访问的外部变量。若试图使用外部变量而没有将其列为捕获,会得到一个编译错误:

\begin{lstlisting}[style=styleCXX]
const char * greeting{ "Hello\n" };
const auto la = []{ return greeting; };
cout << la();
\end{lstlisting}

当尝试用GCC编译这个时,得到了以下的错误:

\begin{tcblisting}{commandshell={}}
In lambda function:
error: 'greeting' is not captured
\end{tcblisting}

因为lambda的主体有自己的词法作用域,而greeting变量在该作用域之外。

可以在捕获中指定greeting变量,这允许变量进入lambda的作用域:

\begin{lstlisting}[style=styleCXX]
const char * greeting{ "Hello\n" };
const auto la = [greeting]{ return greeting; };
cout << la();
\end{lstlisting}

现在程序就能按预期编译和运行:

\begin{tcblisting}{commandshell={}}
$ ./working
Hello
\end{tcblisting}

这种捕获自身作用域之外变量的能力使lambda成为闭包。人们用不同的方式使用这个词,只要能相互理解就好。尽管如此,了解这个术语的含义还是很有必要的。

lambda表达式允许我们编写良好、干净的泛型代码,了可以使用函数式编程模式,其中可以使用lambda作为算法,甚至作为其他lambda函数的参数。

本章中,我们将在以下主题中介绍lambda在STL中的使用:

\begin{itemize}
\item 
用于作用域可重用代码

\item 
算法库中作为谓词

\item 
与std::function一起作为多态包装器

\item 
用递归连接lambda

\item 
将谓词与逻辑连接词结合起来

\item 
用相同的输入调用多个lambda

\item 
对跳转表使用映射lambda
\end{itemize}





