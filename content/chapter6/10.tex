
merge()算法接受两个已排序的序列,并创建第三个已合并并排序的序列。这种技术通常用作归并排序的一部分,允许将非常大量的数据分解成块,分别排序,并合并到一个排序的目标中。

\subsubsection{How to do it…}

对于这个示例,我们将使用std::merge()将两个已排序的vector容器,合并至第三个vector。

\begin{itemize}
\item 
从一个简单的函数开始打印容器的内容:

\begin{lstlisting}[style=styleCXX]
void printc(const auto& c, string_view s = "") {
	if(s.size()) cout << format("{}: ", s);
	for(auto e : c) cout << format("{} ", e);
	cout << '\n';
}
\end{lstlisting}

我们将使用它来打印源序列和目标序列。

\item 
main()函数中,将声明源vector和目标vector,并将它们打印出来:

\begin{lstlisting}[style=styleCXX]
int main() {
	vector<string> vs1{ "dog", "cat",
		"velociraptor" };
	vector<string> vs2{ "kirk", "sulu", "spock" };
	vector<string> dest{};
	printc(vs1, "vs1");
	printc(vs2, "vs2");
	...
}
\end{lstlisting}

输出为:

\begin{tcblisting}{commandshell={}}
vs1: dog cat velociraptor
vs2: kirk sulu spock
\end{tcblisting}

\item 
现在可以对vector进行排序并再次打印:

\begin{lstlisting}[style=styleCXX]
sort(vs1.begin(), vs1.end());
sort(vs2.begin(), vs2.end());
printc(vs1, "vs1 sorted");
printc(vs2, "vs2 sorted");
\end{lstlisting}

输出:

\begin{tcblisting}{commandshell={}}
vs1 sorted: cat dog velociraptor
vs2 sorted: kirk spock sulu
\end{tcblisting}

\item 
现在源容器已经排序,可以将其合并:

\begin{lstlisting}[style=styleCXX]
merge(vs1.begin(), vs1.end(), vs2.begin(), vs2.end(),
	back_inserter(dest));
printc(dest, "dest");
\end{lstlisting}

输出:

\begin{tcblisting}{commandshell={}}
dest: cat dog kirk spock sulu velociraptor
\end{tcblisting}

该输出表示将两个源合并为一个排序的vector。
\end{itemize}

\subsubsection{How it works…}

merge()算法使用begin()和end()迭代器,分别来自源迭代器和目标迭代器的输出迭代器:

\begin{lstlisting}[style=styleCXX]
OutputIt merge(InputIt1, InputIt1, InputIt2, InputIt2, OutputIt)
\end{lstlisting}

其接受两个输入范围,执行归并/排序操作,并将结果序列发送给输出迭代器。

















