
string类是basic\_string类的别名,签名为:

\begin{lstlisting}[style=styleCXX]
class basic_string<char, std::char_traits<char>>;
\end{lstlisting}

第一个模板参数提供字符类型。第二个模板形参提供一个字符特征类,为指定的字符类型提供基本的字符和字符串操作。我们通常使用的是char\_traits<char>类。

我们可以通过提供自己的自定义字符特征类,来修改字符串的行为。

\subsubsection{How to do it…}

在这个示例中,将创建一个字符特征类,用于basic\_string,其可以进行忽略大小的比较:

\begin{itemize}
\item 
首先,需要一个函数将字符转换为通用大小写。这里我们将使用小写字母,但这是一个随意的选择。大写的也可以:

\begin{lstlisting}[style=styleCXX]
constexpr char char_lower(const char& c) {
	if(c >= 'A' && c <= 'Z') return c + ('a' - 'A');
	else return c;
}
\end{lstlisting}

这个函数必须是constexpr(对于C++20及更高版本),所以现有的std::tolower()函数在这里不能工作。

\item 
我们的特征类称为ci\_traits(ci代表大小写无关),继承自std::char\_traits<char>:

\begin{lstlisting}[style=styleCXX]
class ci_traits : public std::char_traits<char> {
	public:
	...
};
\end{lstlisting}

继承允许我们只重载需要的函数。

\item 
比较函数在小于时称为lt(),在等于时称为eq():

\begin{lstlisting}[style=styleCXX]
static constexpr bool lt(char_type a, char_type b)
noexcept {
	return char_lower(a) < char_lower(b);
}
static constexpr bool eq(char_type a, char_type b)
noexcept {
	return char_lower(a) == char_lower(b);
}
\end{lstlisting}

注意,这里比较了字符的小写版本。

\item 
还有一个compare()函数,用于比较两个c字符串。大于则返回+1,小于则返回-1,等于则返回0。我们可以使用三向比较操作符<=>:

\begin{lstlisting}[style=styleCXX]
static constexpr int compare(const char_type* s1,
const char_type* s2, size_t count) {
	for(size_t i{0}; i < count; ++i) {
		auto diff{ char_lower(s1[i]) <=>
			char_lower(s2[i]) };
		if(diff > 0) return 1;
		if(diff < 0) return -1;
	}
	return 0;
}
\end{lstlisting}

\item 
最后,需要实现一个find()函数。这将返回一个指向已找到字符的第一个实例的指针,若未找到则返回nullptr:

\begin{lstlisting}[style=styleCXX]
static constexpr const char_type* find(const char_type*
p, size_t count, const char_type& ch) {
	const char_type find_c{ char_lower(ch) };
	for(size_t i{0}; i < count; ++i) {
		if(find_c == char_lower(p[i])) return p + i;
	}
	return nullptr;
}
\end{lstlisting}

\item 
现在有了一个ci\_traits类,可以为我们的字符串类定义一个别名:

\begin{lstlisting}[style=styleCXX]
using ci_string = std::basic_string<char, ci_traits>;
\end{lstlisting}

\item 
main()函数中,定义了一个字符串和一个ci\_string:

\begin{lstlisting}[style=styleCXX]
int main() {
	string s{"Foo Bar Baz"};
	ci_string ci_s{"Foo Bar Baz"};
	...
\end{lstlisting}

\item 
想用cout打印它们,但这行不通:

\begin{lstlisting}[style=styleCXX]
cout << "string: " << s << '\n';
cout << "ci_string: " << ci_s << '\n';
\end{lstlisting}

首先,需要重载操作符<{}<:

\begin{lstlisting}[style=styleCXX]
std::ostream& operator<<(std::ostream& os,
const ci_string& str) {
	return os << str.c_str();
}
\end{lstlisting}

现在,会得到这样的输出:

\begin{tcblisting}{commandshell={}}
string: Foo Bar Baz
ci_string: Foo Bar Baz
\end{tcblisting}

\item 
比较两个具有不同情况的ci\_string对象:

\begin{lstlisting}[style=styleCXX]
ci_string compare1{"CoMpArE StRiNg"};
ci_string compare2{"compare string"};
if (compare1 == compare2) {
	cout << format("Match! {} == {}\n", compare1,
	compare2);
} else {
	cout << format("no match {} != {}\n", compare1,
	compare2);
}
\end{lstlisting}

输出为:

\begin{tcblisting}{commandshell={}}
Match! CoMpArE StRiNg == compare string
\end{tcblisting}

比较和预期的一样。

\item 
在ci\_s对象上使用find()函数,搜索小写的b并找到大写的B:

\begin{lstlisting}[style=styleCXX]
size_t found = ci_s.find('b');
cout << format("found: pos {} char {}\n", found,
ci_s[found]);
\end{lstlisting}

输出为:

\begin{tcblisting}{commandshell={}}
found: pos 4 char B
\end{tcblisting}
\end{itemize}

\begin{tcolorbox}[colback=webgreen!5!white,colframe=webgreen!75!black,title=Note]
注意format()函数不需要特化,这是用fmt.dev参考实现测试的。它不能与MSVC的format()预览版一起工作,即使是特化也不行。希望在未来的版本中可以修复这个问题。
\end{tcolorbox}

\subsubsection{How it works…}

这个配方的工作方式是将字符串类的模板特化中的std::char\_traits类替换为我们的ci\_traits类。basic\_string类使用特征类来实现其基本的特定于字符的函数,比如比较和搜索。当用我们自己的类来代替它时,可以改变这些基本行为。

\subsubsection{There's more…}

也可以重写assign()和copy()成员函数来创建一个存储小写字符的类:

\begin{lstlisting}[style=styleCXX]
class lc_traits : public std::char_traits<char> {
	public:
	static constexpr void assign( char_type& r, const
	char_type& a )
	noexcept {
		r = char_lower(a);
	}
	static constexpr char_type* assign( char_type* p,
	std::size_t count, char_type a ) {
		for(size_t i{}; i < count; ++i) p[i] =
		char_lower(a);
		return p;
	}
	static constexpr char_type* copy(char_type* dest,
		const char_type* src, size_t count) {
		for(size_t i{0}; i < count; ++i) {
			dest[i] = char_lower(src[i]);
		}
		return dest;
	}
};
\end{lstlisting}

现在,可以创建一个lc\_string别名,对象存储小写字符:

\begin{lstlisting}[style=styleCXX]
using lc_string = std::basic_string<char, lc_traits>;
...
lc_string lc_s{"Foo Bar Baz"};
cout << "lc_string: " << lc_s << '\n';
\end{lstlisting}

输出为:

\begin{tcblisting}{commandshell={}}
lc_string: foo bar baz
\end{tcblisting}

\begin{tcolorbox}[colback=webgreen!5!white,colframe=webgreen!75!black,title=Note]
这些方法在GCC和Clang上可以正常工作,但在MSVC预览版上则不行。我希望在未来的版本中可以修复这个问题。
\end{tcolorbox}



















