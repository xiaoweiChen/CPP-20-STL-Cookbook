
string\_view类提供了string类的轻量级替代方案。string\_view不维护自己的数据存储,而是对C-string的视图进行操作。这使得string\_view比std::string更小,更高效。它在需要字符串对象,但不需要更多内存,对于计算密集型的std::string很有用。

\subsubsection{How to do it…}

string\_view类看起来与STL string类相似,但工作方式略有不同。来看一些例子:

\begin{itemize}
\item 
这是一个从C-string (char数组)初始化的STL字符串:

\begin{lstlisting}[style=styleCXX]
char text[]{ "hello" };
string greeting{ text };
text[0] = 'J';
cout << text << ' ' << greeting << '\n';
\end{lstlisting}

输出:

\begin{tcblisting}{commandshell={}}
Jello hello
\end{tcblisting}

当修改数组时,字符串不会改变。这是因为string构造函数创建了自己的底层数据副本。

\item 
当对string\_view执行同样的操作时,会得到不同的结果:

\begin{lstlisting}[style=styleCXX]
char text[]{ "hello" };
string_view greeting{ text };
text[0] = 'J';
cout << text << ' ' << greeting << '\n';
\end{lstlisting}

输出:

\begin{tcblisting}{commandshell={}}
Jello Jello
\end{tcblisting}

string\_view构造函数创建底层数据的视图,但不保留自己的副本。这会更高效,但也会有副作用。

\item 
因为string\_view不复制底层数据,源数据必须在string\_view对象存在期间保持在作用域内。所以,这是不行的:

\begin{lstlisting}[style=styleCXX]
string_view sv() {
	const char text[]{ "hello" }; // temporary storage
	string_view greeting{ text };
	return greeting;
}
int main() {
	string_view greeting = sv(); // data out of scope
	cout << greeting << '\n'; // output undefined
}
\end{lstlisting}

因为在sv()函数返回后底层数据超出了作用域,所以main()中greeting对象使用它时不再有效。

\item 
string\_view类具有对底层数据有意义的构造函数,包括字符数组(const char*)、连续范围(包括std::string)和其他string\_view对象。下面的例子使用了范围构造函数:

\begin{lstlisting}[style=styleCXX]
string str{ "hello" };
string_view greeting{ str };
cout << greeting << '\n';
\end{lstlisting}

输出:

\begin{tcblisting}{commandshell={}}
hello
\end{tcblisting}

\item 
还有一个stringview文字操作符sv,定义在std::literals命名空间中:

\begin{lstlisting}[style=styleCXX]
using namespace std::literals;
cout << "hello"sv.substr(1, 4) << '\n';
\end{lstlisting}

这构造了一个constexpr的string\_view对象,并调用substr()来获得从索引1开始的4个值。

输出:

\begin{tcblisting}{commandshell={}}
ello
\end{tcblisting}
\end{itemize}

\subsubsection{How it works…}

string\_view类实际上是一个连续字符序列上的迭代器适配器。实现通常有两个成员:一个const CharT *和一个size\_t。工作原理是在源数据周围包装一个contiguous\_iterator。

可以像std::string那样使用它,但有一些区别:

\begin{itemize}
\item 
复制构造函数不复制数据,当复制一个string\_view时,复制操作都会对相同的底层数据进行:

\begin{lstlisting}[style=styleCXX]
char text[]{ "hello" };
string_view sv1{ text };
string_view sv2{ sv1 };
string_view sv3{ sv2 };
string_view sv4{ sv3 };
cout << format("{} {} {} {}\n", sv1, sv2, sv3, sv4);
text[0] = 'J';
cout << format("{} {} {} {}\n", sv1, sv2, sv3, sv4);
\end{lstlisting}

输出:

\begin{tcblisting}{commandshell={}}
hello hello hello hello
Jello Jello Jello Jello
\end{tcblisting}

\item 
记住,将string\_view传递给函数时,它会使用复制构造函数:

\begin{lstlisting}[style=styleCXX]
void f(string_view sv) {
	if(sv.size()) {
		char* x = (char*)sv.data(); // dangerous
		x[0] = 'J'; // modifies the source
	}
	cout << format("f(sv): {} {}\n", (void*)sv.data(),
		sv);
}
int main() {
	char text[]{ "hello" };
	string_view sv1{ text };
	cout << format("sv1: {} {}\n", (void*)sv1.data(),
		sv1);
	f(sv1);
	cout << format("sv1: {} {}\n", (void*)sv1.data(),
		sv1);
}
\end{lstlisting}

输出:

\begin{tcblisting}{commandshell={}}
sv1: 0x7ffd80fa7b2a hello
f(sv): 0x7ffd80fa7b2a Jello
sv1: 0x7ffd80fa7b2a Jello
\end{tcblisting}

因为复制构造函数不会对底层数据进行复制,底层数据的地址(由data()成员函数返回)对于string\_view的所有实例都相同。尽管string\_view成员指针是const限定的,但仍然可以取消const限定符。但不建议这样做,因为可能会导致意想不到的副作用。但需要注意的是,数据永远不会进行复制。

\item 
string\_view类缺少直接操作底层字符串的方法。在string中支持的append()、operator+()、push\_back()、pop\_back()、replace()和resize()等方法在string\_view中不支持。

若需要用+操作符连接字符串,需要使用std::string。例如,这对string\_view不起作用:

\begin{lstlisting}[style=styleCXX]
sv1 = sv2 + sv3 + sv4; // does not work
\end{lstlisting}

这时需要使用string代替:

\begin{lstlisting}[style=styleCXX]
string str1{ text };
string str2{ str1 };
string str3{ str2 };
string str4{ str3 };

str1 = str2 + str3 + str4; // works
cout << str1 << '\n';
\end{lstlisting}

输出:

\begin{tcblisting}{commandshell={}}
JelloJelloJello
\end{tcblisting}
\end{itemize}

