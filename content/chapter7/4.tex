
在C++中,有几种方法可以连接字符串。本节中,我们将研究三个最常见的:字符串类操作符+()、字符串类append()函数和ostringstream类操作符<{}<()。C++20中新增了format()函数。

\subsubsection{How to do it…}

本示例中,将研究连接字符串的方法。然后,执行一些基准测试,并考虑不同的用例。

\begin{itemize}
\item 
从两个std::string对象开始:

\begin{lstlisting}[style=styleCXX]
string a{ "a" };
string b{ "b" };
\end{lstlisting}

string对象是由字面C-string构造的。

C-string构造函数生成字面值字符串的副本,并使用本地副本作为字符串对象的底层数据。

\item 
现在,构造一个新的空字符串对象,并用分隔符和换行符连接a和b:

\begin{lstlisting}[style=styleCXX]
string x{};
x += a + ", " + b + "\n";
cout << x;
\end{lstlisting}

这里,使用字符串对象的+=和+操作符来连接a和b字符串,以及字面字符串“,”和“\verb|\|n”。结果字符串包含连接在一起的元素:

\begin{tcblisting}{commandshell={}}
a, b
\end{tcblisting}

\item 
可以使用string对象的append()成员函数:

\begin{lstlisting}[style=styleCXX]
string x{};
x.append(a);
x.append(", ");
x.append(b);
x.append("\n");
cout << x;
\end{lstlisting}

结果相同:

\begin{tcblisting}{commandshell={}}
a, b
\end{tcblisting}

\item 
或者,可以构造一个ostringstream对象,使用流接口:

\begin{lstlisting}[style=styleCXX]
ostringstream x{};
x << a << ", " << b << "\n";
cout << x.str();
\end{lstlisting}

我们得到了相同的结果:

\begin{tcblisting}{commandshell={}}
a, b
\end{tcblisting}

\item 
也可以使用C++20的format()函数:

\begin{lstlisting}[style=styleCXX]
string x{};
x = format("{}, {}\n", a, b);
cout << x;
\end{lstlisting}

还是相同的结果:

\begin{tcblisting}{commandshell={}}
a, b
\end{tcblisting}

\end{itemize}

\subsubsection{How it works…}

string对象有两种不同的方法用于连接字符串:+操作符和append()成员函数。

append()成员函数的作用是:将数据添加到字符串对象的数据末尾。必须分配和管理内存来完成这个任务。

+操作符使用操作符+()重载构造一个包含新旧数据的新字符串对象,并返回新对象。

ostringstream对象的工作方式类似于ostream,但将其输出存储为字符串使用。

C++20的format()函数使用带有可变参数的格式字符串,并返回一个新构造的字符串对象。

\subsubsection{There's more…}

如何决定哪种连接策略适合自己的代码?可以从一些基准测试开始。

\noindent
\textbf{基准测试}

我在Debian Linux上使用GCC 11执行了这些测试:

\begin{itemize}
\item 
首先,使用<chrono>库创建一个计时器函数:

\begin{lstlisting}[style=styleCXX]
using std::chrono::high_resolution_clock;
using std::chrono::duration;

void timer(string(*f)()) {
	auto t1 = high_resolution_clock::now();
	string s{ f() };
	auto t2 = high_resolution_clock::now();
	duration<double, std::milli> ms = t2 - t1;
	cout << s;
	cout << format("duration: {} ms\n", ms.count());
}
\end{lstlisting}

timer函数调用传递给它的函数,标记函数调用前后的时间。然后使用cout显示持续时间。

\item 
现在,使用append()成员函数创建了一个连接字符串的函数:

\begin{lstlisting}[style=styleCXX]
string append_string() {
	cout << "append_string\n";
	string a{ "a" };
	string b{ "b" };
	long n{0};
	while(++n) {
		string x{};
		x.append(a);
		x.append(", ");
		x.append(b);
		x.append("\n");
		if(n >= 10000000) return x;
	}
	return "error\n";
}
\end{lstlisting}

为了进行基准测试,该函数将重复连接1000万次。使用timer()从main()调用这个函数:

\begin{lstlisting}[style=styleCXX]
int main() {
	timer(append_string);
}
\end{lstlisting}

得到这样的输出:

\begin{tcblisting}{commandshell={}}
append_string
a, b
duration: 425.361643 ms
\end{tcblisting}

所以,在这个系统上,连接字符串在大约425毫秒内运行了1000万次。

\item 
现在,用+操作符重载创建相同的函数:

\begin{lstlisting}[style=styleCXX]
string concat_string() {
	cout << "concat_string\n";
	string a{ "a" };
	string b{ "b" };
	long n{0};
	while(++n) {
		string x{};
		x += a + ", " + b + "\n";
		if(n >= 10000000) return x;
	}
	return "error\n";
}
\end{lstlisting}

基准测试输出:

\begin{tcblisting}{commandshell={}}
concat_string
a, b
duration: 659.957702 ms
\end{tcblisting}

这个版本在大约660毫秒内执行了1000万次。

\item 
现在,让用ostringstream试试:

\begin{lstlisting}[style=styleCXX]
string concat_ostringstream() {
	cout << "ostringstream\n";
	string a { "a" };
	string b { "b" };
	long n{0};
	while(++n) {
		ostringstream x{};
		x << a << ", " << b << "\n";
		if(n >= 10000000) return x.str();
	}
	return "error\n";
}
\end{lstlisting}

基准测试输出:

\begin{tcblisting}{commandshell={}}
ostringstream
a, b
duration: 3462.020587 ms
\end{tcblisting}

这个版本在3.5秒内运行了1000万次迭代。

\item 
下面是format()版本:

\begin{lstlisting}[style=styleCXX]
string concat_format() {
	cout << "append_format\n";
	string a{ "a" };
	string b{ "b" };
	long n{0};
	while(++n) {
		string x{};
		x = format("{}, {}\n", a, b);
		if(n >= 10000000) return x;
	}
	return "error\n";
}
\end{lstlisting}

基准测试输出:

\begin{tcblisting}{commandshell={}}
append_format
a, b
duration: 782.800547 ms
\end{tcblisting}

format()版本在大约783毫秒内运行了1000万次迭代。

\item 
结果总结:

% Please add the following required packages to your document preamble:
% \usepackage[table,xcdraw]{xcolor}
% If you use beamer only pass "xcolor=table" option, i.e. \documentclass[xcolor=table]{beamer}
\begin{table}[H]
\centering
\begin{tabular}{|c|c|}
	\hline
	\rowcolor[HTML]{9B9B9B} 
	{\color[HTML]{FFFFFF} 连接方式} & {\color[HTML]{FFFFFF} 基准(毫秒)} \\ \hline
	append()      & 425 ms   \\ \hline
	operator+()   & 660 ms   \\ \hline
	format()      & 783 ms   \\ \hline
	ostringstream & 3,462 ms \\ \hline
\end{tabular}

\hspace*{\fill} \\ %插入空行
连接字符串的性能比较
\end{table}

\end{itemize}

\noindent
\textbf{为什么会出现性能差异?}

可以从这些基准测试中看到,ostringstream版本比基于字符串的版本花费的时间长很多倍。

append()方法比+运算符略快,需要分配内存,但不构造新对象。由于重复进行,所以内部可能有一些优化。

+操作符重载可能调用append()方法,所以函数调用会使它比append()方法慢。

format()版本创建了一个新的字符串对象,但没有iostream的开销。

ostringstream操作符<{}<重载为每个操作创建一个新的ostream对象。考虑到流对象的复杂性,以及对流状态的管理,这使得它比基于字符串其他版本都要慢得多。

\noindent
\textbf{如何选择?}

其中会涉及到一些个人偏好的因素。操作符重载(+或<{}<)可以很方便,若不考虑性能的话。

与string方法相比,ostringstream类有一个明显的优势:针对每种不同的类型,有相应的<{}<操作符,因此能够在不同类型调用相同代码的情况下进行操作。

format()函数提供了相同的类型安全和自定义选项,并且比ostringstream类快得多。

string对象的+操作符重载快速、易于使用、易于读取,但比append()慢了一点点。

append()版本是最快的,但需要为每个项使用单独的函数。

通常,我更喜欢format()函数或字符串对象的+运算符。如果性能很重要,我会使用append()。我将在需要ostringstream独特特性,且性能不是问题的地方使用它。