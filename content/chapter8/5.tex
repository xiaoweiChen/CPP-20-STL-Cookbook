
std::chrono库提供了用于测量、报告时间和间隔的工具。

这些类和函数中的许多都是在C++11中引入的,但在C++20中已经有了重大的变化和更新。在撰写本文时,许多更新还没有在我测试的系统上实现。

本示例使用chrono库,为事件进行计时。

\subsubsection{How to do it…}

system\_clock类用于报告当前日期和时间。steady\_clock和high\_resolution\_clock类用于计时事件。让我们来看看这些时钟间的区别:

\begin{itemize}
\item 
因为这些名字可能又长又不方便,在整个示例中使用一些类型别名:

\begin{lstlisting}[style=styleCXX]
using std::chrono::system_clock;
using std::chrono::steady_clock;
using std::chrono::high_resolution_clock;
using std::chrono::duration;
using seconds = duration<double>;
using milliseconds = duration<double, std::milli>;
using microseconds = duration<double, std::micro>;
using fps24 = duration<unsigned long, std::ratio<1, 24>>;
\end{lstlisting}

duration类表示两个时间点之间的间隔。这些别名便于使用不同的间隔。

\item 
可以使用system\_clock类来获取当前时间和日期:

\begin{lstlisting}[style=styleCXX]
auto t = system_clock::now();
cout << format("system_clock::now is {:%F %T}\n", t);
\end{lstlisting}

system\_clock::now()函数的作用是:返回一个time\_point对象。<chrono>库包含了time\_point的format()特化,该特化使用strftime()格式说明符。

输出为:

\begin{tcblisting}{commandshell={}}
system_clock::now is 2022-02-05 13:52:15
\end{tcblisting}

<iomanip>头文件包括put\_time(),其工作方式类似于ostream的strftime():

\begin{lstlisting}[style=styleCXX]
std::time_t now_t = system_clock::to_time_t(t);
cout << "system_clock::now is "
	<< std::put_time(std::localtime(&now_t), "%F %T")
	<< '\n';
\end{lstlisting}

put\_time()接受一个指向C风格time\_t*值的指针。system\_clock::to\_time\_t转换time\_point对象为time\_t。

这与format()示例的输出相同:

\begin{tcblisting}{commandshell={}}
system_clock::now is 2022-02-05 13:52:15
\end{tcblisting}

\item 
也可以使用system\_clock为事件计时。首先,需要进行计时。下面是一个计算质数的函数:

\begin{lstlisting}[style=styleCXX]
constexpr uint64_t MAX_PRIME{ 0x1FFFF }
uint64_t count_primes() {
	constexpr auto is_prime = [](const uint64_t n) {
		for(uint64_t i{ 2 }; i < n / 2; ++i) {
			if(n % i == 0) return false;
		}
		return true;
	};
	uint64_t count{ 0 };
	uint64_t start{ 2 };
	uint64_t end{ MAX_PRIME };
	for(uint64_t i{ start }; i <= end ; ++i) {
		if(is_prime(i)) ++count;
	}
	return count;
}
\end{lstlisting}

这个函数对2到0x1FFFF(131,071)之间的质数进行计数,在大多数现代系统上,这需要几秒钟的时间。

\item 
现在,编写一个计时器函数来计时count\_primes():

\begin{lstlisting}[style=styleCXX]
seconds timer(uint64_t(*f)()) {
	auto t1{ system_clock::now() };
	uint64_t count{ f() };
	auto t2{ system_clock::now() };
	seconds secs{ t2 - t1 };
	cout << format("there are {} primes in range\n",
		count);
	return secs;
}
\end{lstlisting}

这个函数接受一个函数f,并返回duration<double>。这里使用system\_clock::now()来标记调用f()前后的时间。取两个时间之间的差值,并在duration对象中返回。

\item 
可以在main()中使用timer():

\begin{lstlisting}[style=styleCXX]
int main() {
	auto secs{ timer(count_primes) };
	cout << format("time elapsed: {:.3f} seconds\n",
	secs.count());
	...
\end{lstlisting}

这将把count\_primes()函数传递给timer(),并以秒为单位存储持续时间对象。

输出为:

\begin{tcblisting}{commandshell={}}
there are 12252 primes in range
time elapsed: 3.573 seconds
\end{tcblisting}

duration对象上的count()方法返回指定单位的持续时间——本例中为double,表示持续时间的秒数。

这是在一个运行Debian和GCC的VM上运行的,具体的时间在不同的系统上会有所不同。

\item 
system\_clock类设计用来提供当前挂钟时间。虽然其频率可能支持计时目的,但不保证单调。换句话说,它可能不能提供一致的计时间隔。

chrono库在steady\_clock中提供了一个更合适的时钟。它具有与system\_clock相同的接口,但为计时目的提供了更可靠的计时间隔:

\begin{lstlisting}[style=styleCXX]
seconds timer(uint64_t(*f)()) {
	auto t1{ steady_clock::now() };
	uint64_t count{ f() };
	auto t2{ steady_clock::now() };
	seconds secs{ t2 - t1 };
	cout << format("there are {} primes in range\n",
		count);
	return secs;
}
\end{lstlisting}

steady\_clock设计用于提供可靠一致的单调,适用于计时事件。它使用一个相对的时间参考,所以不用挂钟时间。system\_clock从固定的时间点(1970年1月1日00:00 UTC)开始测量,而steady\_clock使用相对时间。

另一个选项是high\_resolution\_clock,在给定系统上提供最短的计时间隔期,但在不同的实现中实现不一致。其可能是system\_clock或steady\_clock的别名,可能是单调的,也可能不是。

high\_resolution\_clock不建议通用。

\item 
timer()函数返回seconds,是duration<double>的别名:

\begin{lstlisting}[style=styleCXX]
using seconds = duration<double>;
\end{lstlisting}

duration类接受一个可选的第二个模板形参,一个std::ratio类:

\begin{lstlisting}[style=styleCXX]
template<class Rep, class Period = std::ratio<1>>
class duration;
\end{lstlisting}

<chrono>头文件为许多十进制比率提供了方便的类型,包括milli和micro:

\begin{lstlisting}[style=styleCXX]
using milliseconds = duration<double, std::milli>;
using microseconds = duration<double, std::micro>;
\end{lstlisting}

若需要其他工具,可以自己准备:

\begin{lstlisting}[style=styleCXX]
using fps24 = duration<unsigned long, std::ratio<1, 24>>;
\end{lstlisting}

fps24表示以标准的每秒24帧拍摄的电影帧数。比率是1/24秒。

这让我们可以轻松地在不同的持续时间范围之间进行转换:

\begin{lstlisting}[style=styleCXX]
cout << format("time elapsed: {:.3f} sec\n", secs.
count());
cout << format("time elapsed: {:.3f} ms\n",
	milliseconds(secs).count());
cout << format("time elapsed: {:.3e} μs\n",
	microseconds(secs).count());
cout << format("time elapsed: {} frames at 24 fps\n",
	floor<fps24>(secs).count());
\end{lstlisting}

输出为:

\begin{tcblisting}{commandshell={}}
time elapsed: 3.573 sec
time elapsed: 3573.077 ms
time elapsed: 3.573e+06 μs
time elapsed: 85 frames at 24 fps
\end{tcblisting}

因为fps24别名使用unsigned long,而不是double,所以需要进行类型转换。floor函数通过丢弃小数部分来实现这一点。Round()和ceil()在此上下文中也可用。

\item 
方便起见,chrono库提供了标准持续时间比的format()特化:

\begin{lstlisting}[style=styleCXX]
cout << format("time elapsed: {:.3}\n", secs);
cout << format("time elapsed: {:.3}\n",
milliseconds(secs));
cout << format("time elapsed: {:.3}\n",
microseconds(secs));
\end{lstlisting}

输出为:

\begin{tcblisting}{commandshell={}}
time elapsed: 3.573s
time elapsed: 3573.077ms
time elapsed: 3573076.564μs
\end{tcblisting}

这些结果在不同的实现上有所不同。
\end{itemize}

\subsubsection{How it works…}

chrono库有两个主要部分,时钟类和间隔类。

\noindent
\textbf{时钟类}

时钟类包括:

\begin{itemize}
\item 
system\_clock – 提供挂钟时间。

\item 
steady\_clock – 为时间测量提供有保证的单调计时间隔。

\item 
high\_resolution\_clock – 提供最短的可用计时间隔周期。某些系统上,其可能是system\_clock或steady\_clock的别名。

\end{itemize}

我们使用system\_clock显示当前时间和日期,用steady\_clock来测量间隔。

每个时钟类都有一个now()方法,该方法返回time\_point,表示时钟的当前值。now()是一个静态成员函数,所以可以不实例化对象:

\begin{lstlisting}[style=styleCXX]
auto t1{ steady_clock::now() };
\end{lstlisting}

\noindent
\textbf{std::duration类}

duration类用于保存时间间隔——即两个time\_point对象之间的差值。它通常由time\_point对象的减法(-)操作符构造。

\begin{lstlisting}[style=styleCXX]
duration<double> secs{ t2 - t1 };
\end{lstlisting}

time\_point减去操作符作为duration的构造函数:

\begin{lstlisting}[style=styleCXX]
template<class C, class D1, class D2>
constexpr duration<D1,D2>
operator-( const time_point<C,D1>& pt_lhs,
	const time_point<C,D2>& pt_rhs );
\end{lstlisting}

duration类有用于类型表示的模板参数和一个ratio对象:

\begin{lstlisting}[style=styleCXX]
template<class Rep, class Period = std::ratio<1>>
class duration;
\end{lstlisting}

周期模板参数默认为1:1,即秒。

标准库为10次方提供了从atto(1/1,000,000,000,000,000,000)到exa(1,000,000,000,000,000,000/1)的比率别名(例如micro和milli)。我们可以创建标准的持续时间,就像示例中那样:

\begin{lstlisting}[style=styleCXX]
using milliseconds = duration<double, std::milli>;
using microseconds = duration<double, std::micro>;
\end{lstlisting}

count()方法给出了Rep类型中的持续时间:

\begin{lstlisting}[style=styleCXX]
constexpr Rep count() const;
\end{lstlisting}

这使得我们可以方便地访问持续时间,并用于显示或其他目的:

\begin{lstlisting}[style=styleCXX]
cout << format("duration: {}\n", secs.count());
\end{lstlisting}
















