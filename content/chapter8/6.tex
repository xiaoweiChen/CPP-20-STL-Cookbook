
tuple类本质上是一个更复杂、不方便的结构体。tuple的接口很麻烦,类模板参数推导和结构化绑定可以使这个过程变得简单。

对于大多数应用程序,我倾向于在使用tuple之前使用struct,但有一个例外:tuple的一个真正优势是,可以在可变的上下文中与折叠表达式一起使用。

\subsubsection{折叠表达式}

折叠表达式是C++17的一个新特性,设计目的是为了更容易地展开可变参数包。在折叠表达式之前,展开参数包需要一个递归函数:

\begin{lstlisting}[style=styleCXX]
template<typename T>
void f(T final) {
	cout << final << '\n';
}
template<typename T, typename... Args>
void f(T first, Args... args) {
	cout << first;
	f(args...);
}
int main() {
	f("hello", ' ', 47, ' ', "world");
}
\end{lstlisting}

输出为:

\begin{tcblisting}{commandshell={}}
hello 47 world
\end{tcblisting}

使用折叠表达式后,这就会简单许多了:

\begin{lstlisting}[style=styleCXX]
template<typename... Args>
void f(Args... args) {
	(cout << ... << args);
	cout << '\n';
}
\end{lstlisting}

输出为:

\begin{tcblisting}{commandshell={}}
hello 47 world
\end{tcblisting}

折叠表达式有四种类型:

\begin{itemize}
\item 
一元右折叠: (args op ...)

\item 
一元左折叠: (... op args)

\item 
二元右折叠: (args op ... op init)

\item 
二元左折叠: (init op ... op args)
\end{itemize}

上面例子中的表达式是一个二进制左折叠:

\begin{lstlisting}[style=styleCXX]
(cout << ... << args);
\end{lstlisting}

可扩展为:

\begin{lstlisting}[style=styleCXX]
cout << "hello" << ' ' << 47 << ' ' << "world";
\end{lstlisting}

折叠表达式对于很多目的来说都是非常方便的,来看看如何在tuple中使用。

\subsubsection{How to do it…}

这个示例中,我们将创建一个模板函数,该函数对具有不同数量和类型元素的tuple进行操作:

\begin{itemize}
\item 
这示例的核心是一个函数,其接受一个未知大小和类型的tuple,并使用format()打印每个元素:

\begin{lstlisting}[style=styleCXX]
template<typename... T>
constexpr void print_t(const tuple<T...>& tup) {
	auto lpt =
	[&tup] <size_t... I>
	(std::index_sequence<I...>)
	constexpr {
		(..., ( cout <<
			format((I? ", {}" : "{}"),
				get<I>(tup))
		));
		cout << '\n';
	};
	lpt(std::make_index_sequence<sizeof...(T)>());
}
\end{lstlisting}

这个函数的核心在lambda表达式中,使用index\_sequence对象生成索引值的参数包。然后,使用折叠表达式对每个下标值调用get<I>。模板化的lambda需要编译器支持C++20。

可以使用一个单独的函数来代替lambda,但我更喜欢将它保持在一个作用域中。

\item 
现在可以在main()中使用各种tuple:

\begin{lstlisting}[style=styleCXX]
int main() {
	tuple lables{ "ID", "Name", "Scale" };
	tuple employee{ 123456, "John Doe", 3.7 };
	tuple nums{ 1, 7, "forty-two", 47, 73L, -111.11 };
	
	print_t(lables);
	print_t(employee);
	print_t(nums);
}
\end{lstlisting}

输出为:

\begin{tcblisting}{commandshell={}}
ID, Name, Scale
123456, John Doe, 3.7
1, 7, forty-two, 47, 73, -111.11
\end{tcblisting}
\end{itemize}


\subsubsection{How it works…}

使用tuple的挑战在于它的限制性接口。可以使用std::tie()、结构化绑定或std::get<>函数检索元素。若不知道tuple中元素的数量和类型,这些技术都没啥用。

可以使用index\_sequence类来解决这个限制。index\_sequence是一个integer\_sequence的特化,它提供了一个size\_t的参数包,可以用它来索引tuple。我们使用make\_index\_sequence调用lambda函数在lambda中设置一个参数包:

\begin{lstlisting}[style=styleCXX]
lpt(std::make_index_sequence<sizeof...(T)>());
\end{lstlisting}

模板化的lambda是用get()函数size\_t索引的参数包构造的:

\begin{lstlisting}[style=styleCXX]
[&tup] <size_t... I> (std::index_sequence<I...>) constexpr {
	...
};
\end{lstlisting}

get()函数将索引值作为模板形参,这里可以使用一元左折叠表达式调用get<I>():

\begin{lstlisting}[style=styleCXX]
(..., ( cout << format("{} ", std::get<I>(tup))));
\end{lstlisting}

折叠表达式接受函数参数包中的每个元素,并应用逗号操作符。逗号的右边有一个format()函数,输出元组的每个元素。

这使得推断tuple中元素的数量成为可能,这使得它可以在可变的上下文中使用。记住,与一般的模板函数一样,编译器将为每个tuple形参组合生成该函数的特化版本。


\subsubsection{There's more…}

也可以将这种技术用于其他任务。例如,这里有一个函数,它返回大小未知的元组中所有int值的和:

\begin{lstlisting}[style=styleCXX]
template<typename... T>
constexpr int sum_t(const tuple<T...>& tup) {
	int accum{};
	auto lpt =
	[&tup, &accum] <size_t... I>
		(std::index_sequence<I...>)
	constexpr {
		(..., (
			accum += get<I>(tup)
		));
	};
	lpt(std::make_index_sequence<sizeof...(T)>());
	return accum;
}
\end{lstlisting}

可以用几个不同数量int值的元组对象调用这个函数:

\begin{lstlisting}[style=styleCXX]
tuple ti1{ 1, 2, 3, 4, 5 };
tuple ti2{ 9, 10, 11, 12, 13, 14, 15 };
tuple ti3{ 47, 73, 42 };
auto sum1{ sum_t(ti1) };
auto sum2{ sum_t(ti2) };
auto sum3{ sum_t(ti3) };
cout << format("sum of ti1: {}\n", sum1);
cout << format("sum of ti2: {}\n", sum2);
cout << format("sum of ti3: {}\n", sum3);
\end{lstlisting}

输出为:

\begin{tcblisting}{commandshell={}}
sum of ti1: 15
sum of ti2: 84
sum of ti3: 162
\end{tcblisting}










