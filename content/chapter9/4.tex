
std::async()异步运行一个目标函数,并返回一个携带目标函数返回值的std::future对象。通过这种方式,async()的操作很像std::thread,但允许函数有返回值。

通过几个例子来了解一下std::async()的使用。

\subsubsection{How to do it…}

在其最简单的形式中,std::async()函数执行与std::thread相同的任务,不需要调用join()或detach(),同时还允许通过std::future对象返回值。

这个示例中,将使用一个函数来计算一个范围内质数的数量。可以使用chrono::steady\_clock为每个线程的进行计时。

\begin{itemize}
\item 
先从别名开始:

\begin{lstlisting}[style=styleCXX]
using launch = std::launch;
using secs = std::chrono::duration<double>;
\end{lstlisting}

std::launch有启动策略常量,用于async()调用。secs别名是一个持续时间类,用于为质数计算计时。

\item 
我们的目标函数计算范围内的质数。这本质上是一种通过占用一些时钟周期,来理解执行策略的方法:

\begin{lstlisting}[style=styleCXX]
struct prime_time {
	secs dur{};
	uint64_t count{};
};
prime_time count_primes(const uint64_t& max) {
	prime_time ret{};
	constexpr auto isprime = [](const uint64_t& n) {
		for(uint64_t i{ 2 }; i < n / 2; ++i) {
			if(n % i == 0) return false;
		}
		return true;
	};
	uint64_t start{ 2 };
	uint64_t end{ max };
	auto t1 = steady_clock::now();
	for(uint64_t i{ start }; i <= end ; ++i) {
		if(isprime(i)) ++ret.count;
	}
	ret.dur = steady_clock::now() - t1;
	return ret;
}
\end{lstlisting}

prime\_time结构用于返回值,可用于duration和count,这样就可以计算循环本身的时间。若一个值是质数,isprime返回true。我们使用steady\_clock,来计算质数计数循环的持续时间。

\item 
在main()中,调用函数并报告其运行时间:

\begin{lstlisting}[style=styleCXX]
int main() {
	constexpr uint64_t MAX_PRIME{ 0x1FFFF };
	auto pt = count_primes(MAX_PRIME);
	cout << format("primes: {} {:.3}\n", pt.count,
		pt.dur);
}
\end{lstlisting}

输出为:

\begin{tcblisting}{commandshell={}}
primes: 12252 1.88008s
\end{tcblisting}

\item 
现在,可以用std::async()异步运行count\_primes():

\begin{lstlisting}[style=styleCXX]
int main() {
	constexpr uint64_t MAX_PRIME{ 0x1FFFF };
	auto primes1 = async(count_primes, MAX_PRIME);
	auto pt = primes1.get();
	cout << format("primes: {} {:.3}\n", pt.count,
		pt.dur);
}
\end{lstlisting}

这里,使用count\_primes函数和MAX\_PRIME参数调用async(),这count\_primes函数放在后台运行。

async()返回std::future对象,该对象携带异步操作的返回值。future对象的get()是阻塞式的,会等到异步函数完成,获取该函数的返回对象。

这与没有async()时运行的时间几乎相同:

\begin{tcblisting}{commandshell={}}
primes: 12252 1.97245s
\end{tcblisting}

\item 
async()函数可选地将执行策略标志作为其第一个参数:

\begin{tcblisting}{commandshell={}}
auto primes1 = async(launch::async, count_primes, MAX_
PRIME);
\end{tcblisting}

选择是异步或延迟的,这些标志位于std::launch命名空间中。

async标志启用异步操作,deferred标志启用延迟计算。这些标志是位映设的,可以与按位或|操作符组合使用。

默认情况下,这两个位都要设置,就和async|deferred一样。

\item 
可以使用async()同时运行函数的几个实例:

\begin{lstlisting}[style=styleCXX]
int main() {
	constexpr uint64_t MAX_PRIME{ 0x1FFFF };
	list<std::future<prime_time>> swarm;
	cout << "start parallel primes\n";
	auto t1{ steady_clock::now() };
	for(size_t i{}; i < 15; ++i) {
		swarm.emplace_back(
		async(launch::async, count_primes,
		MAX_PRIME)
		);
	}
	for(auto& f : swarm) {
		static size_t i{};
		auto pt = f.get();
		cout << format("primes({:02}): {} {:.5}\n",
		++i, pt.count, pt.dur);
	}
	secs dur_total{ steady_clock::now() - t1 };
	cout << format("total duration: {:.5}s\n",
		dur_total.count());
}
\end{lstlisting}

我们知道async返回一个future对象。因此,可以通过将未来对象存储在容器中来运行15个线程。以下是我运行在Windows设备(6核i7)上的输出:

\begin{tcblisting}{commandshell={}}
start parallel primes
primes(01): 12252 4.1696s
primes(02): 12252 3.7754s
primes(03): 12252 3.78089s
primes(04): 12252 3.72149s
primes(05): 12252 3.72006s
primes(06): 12252 4.1306s
primes(07): 12252 4.26015s
primes(08): 12252 3.77283s
primes(09): 12252 3.77176s
primes(10): 12252 3.72038s
primes(11): 12252 3.72416s
primes(12): 12252 4.18738s
primes(13): 12252 4.07128s
primes(14): 12252 2.1967s
primes(15): 12252 2.22414s
total duration: 5.9461s
\end{tcblisting}

尽管,6核i7不能在不同的核中运行所有进程,但仍然可以在6秒内完成15个实例。

看起来它在4秒内完成了前13个线程,然后再花2秒完成最后2个线程。这似乎利用了Intel的超线程技术,该技术允许在某些情况下在一个核心中运行2个线程。

当在12核Xeon上运行相同的代码时,会得到这样的结果:

\begin{tcblisting}{commandshell={}}
start parallel primes
primes(01): 12252 0.96221s
primes(02): 12252 0.97346s
primes(03): 12252 0.92189s
primes(04): 12252 0.97499s
primes(05): 12252 0.98135s
primes(06): 12252 0.93426s
primes(07): 12252 0.90294s
primes(08): 12252 0.96307s
primes(09): 12252 0.95015s
primes(10): 12252 0.94255s
primes(11): 12252 0.94971s
primes(12): 12252 0.95639s
primes(13): 12252 0.95938s
primes(14): 12252 0.92115s
primes(15): 12252 0.94122s
total duration: 0.98166s
\end{tcblisting}

12核的至强处理器可以在一秒钟内完成全部15个进程的任务。

\end{itemize}

\subsubsection{How it works…}

理解std::async的关键在于其对std::promise和std::future的使用。

promise类允许线程存储一个对象,以后可以由future对象进行异步检索。

例如,有这样一个函数:

\begin{lstlisting}[style=styleCXX]
void f() {
	cout << "this is f()\n";
}
\end{lstlisting}

可以用std::thread运行:

\begin{lstlisting}[style=styleCXX]
int main() {
	std::thread t1(f);
	t1.join();
	cout << "end of main()\n";
}
\end{lstlisting}

这对于没有返回值的简单函数来说很好。当想从f()返回一个值时,可以使用promise和future。

在main()线程中设置了promise和future对象:

\begin{lstlisting}[style=styleCXX]
int main() {
	std::promise<int> value_promise;
	std::future<int> value_future =
		value_promise.get_future();
	std::thread t1(f, std::move(value_promise));
	t1.detach();
	cout << format("value is {}\n", value_future.get());
	cout << "end of main()\n";
}
\end{lstlisting}

然后,将promise对象传递给函数:

\begin{lstlisting}[style=styleCXX]
void f(std::promise<int> value) {
	cout << "this is f()\n";
	value.set_value(47);
}
\end{lstlisting}

promise对象不能复制,因此需要使用std::move将其传递给函数。

promise对象充当到future对象的桥梁,其允许在值可用的前提下,对其进行检索。

std::async()只是一个辅助函数,用于简化promise和future对象的创建。可以这样使用async():

\begin{lstlisting}[style=styleCXX]
int f() {
	cout << "this is f()\n";
	return 47;
}

int main() {
	auto value_future = std::async(f);
	cout << format("value is {}\n", value_future.get());
	cout << "end of main()\n";
}
\end{lstlisting}

这就是async()的值。在很多情况下,promise与future的组合使用起来更容易。

















